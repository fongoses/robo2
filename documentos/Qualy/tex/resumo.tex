%==========================================================
% Resumo.
% 
% Autor
% Heitor Luis Polidoro
% 
% Orientador
% Prof. Dr. Denis Fernando Wolf
%==========================================================

\pagestyle{abstractst}

\chapter*{Resumo}
\label{resumo}
Um dos principais desafios na �rea de rob�tica m�vel � o desenvolvimento de sistemas aut�nomos, que sejam capazes de interagirem com o ambiente, aprenderem e tomarem decis�es corretas para que suas tarefas sejam executadas com �xito. O desenvolvimento destes sistemas aut�nomos inteligentes consiste em uma �rea de pesquisa multidisciplinar que envolve: intelig�ncia artificial, aprendizado de m�quina, estima��o estat�stica e sistemas embarcados, por exemplo. Dentro desse contexto, esse projeto de pesquisa aborda o problema de navega��o e monitoramento de ambientes utilizando rob�s m�veis. Dado uma representa��o do ambiente (mapa topol�gica) e a prioridade de cada uma das regi�es do mapa, o rob� deve estimar qual o percurso mais eficiente para monitorar esse ambiente. Esse projeto de pesquisa prop�e a o desenvolvimento e compara��o de heur�sticas e algoritmos de monitoramento de ambientes internos de acordo com alguns crit�rios propostos no projeto, levando em considera��o a prioridade de cada local a ser visitado e o tempo decorrido desde a �ltima visita a cada regi�o do ambiente. Entre as diversas aplica��es pr�ticas desse tipo de algoritmo, destacam-se o desenvolvimento de sistemas de seguran�a m�veis inteligentes e sistemas de monitoramento em hospitais. Experimentos com diferentes algoritmos em diferentes ambientes s�o apresentados.




