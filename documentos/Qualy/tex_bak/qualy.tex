\newif\ifpdf
  \ifx\pdfoutput\undefined
  \pdffalse
\else
  \pdftrue
\fi

\documentclass[12pt,twoside]{book}

%%% define a macro \ifpdf para compila��o condicional -- PDF ou
%%% DVI/PS.


\usepackage[ruled]{algorithm2e}
\renewcommand{\listalgorithmcfname}{Lista de Algoritmos}%
\renewcommand{\algorithmcfname}{Algoritmo}%

\usepackage[english,brazil]{babel}
\usepackage{subfigure}

%%% acentua��o
\usepackage[latin1]{inputenc}


%%% bibliografia
%\usepackage{chicago}
\usepackage{natbib}

\usepackage{latexsym}


\usepackage{setspace}


\usepackage{xspace}


\usepackage{acronym}
%%% referencias com p�gina \vref
\usepackage[brazil]{varioref}

%%% identa primeiro par�grafo
\usepackage{indentfirst}

%%% figuras um ao lado do outro
\usepackage{subfigure}


%%% definir t�tulos de se��o
\usepackage[sf,sl,outermarks]{titlesec}


%%% boxes
\usepackage{fancybox}
\usepackage{fancyvrb}
\usepackage{color}

\ifpdf
%%% somente na vers�o PDF

%%% Para inclus�o de gr�ficos
%\usepackage[pdftex]{graphicx}
\usepackage{graphicx}

%%% dimens�es do documento
\usepackage[pdftex]{geometry}
  \geometry{a4paper,left=3cm,right=2cm,top=2.0cm,bottom=2cm,twoside}

%%% cria links no arquivo .pdf
%%% comentar as linhas na vers�o para impressao
\usepackage[pdftex,pdfpagelabels,pagebackref]{hyperref}
%\usepackage{hyperref}


%%% propriedades do arquivo .pdf
\hypersetup{
    pdftitle = {},
    pdfsubject = {},
    pdfkeywords = {},
    pdfauthor = {}
    }



\else
%%% somente na vers�o DVI/PS

%%% Para inclus�o de gr�ficos
\usepackage[dvips]{graphicx}

%%% dimens�es do documento
\usepackage[dvips]{geometry}
 \geometry{a4paper,left=3cm,right=2cm,top=2.0cm,bottom=2cm,twoside}

\usepackage[dvips]{hyperref}

\fi

\hypersetup{colorlinks=true,linkcolor=black,citecolor=black,hypertexnames=false}

%%fonte
\usepackage{bookman}
\usepackage[T1]{fontenc}

%%para os numeros reais
\usepackage{amsfonts}

\DeclareFixedFont{\numberfont}{T1}{phv}{bx}{n}{2cm}

%%% redefine o formato do t�tulo
\titleformat{\chapter}[display]
  {\normalfont\Large\sffamily
  }
  {%\titlerule[3pt]%
   \filright
   \rule[32pt]{.7\linewidth}{4pt}
   \hspace{-11pt}
   \shadowbox{
   \begin{minipage}{.18\linewidth}
     \begin{center}
       \textsc{\Large\chaptertitlename}\\
       \vspace{1ex}
       {\numberfont\color[gray]{0.5} \thechapter}\\
       \vspace{1ex}
     \end{center}
   \end{minipage}}
  }
  {0pt}
  {\filcenter
   \Huge
   }
  [\hfill\rule{.8\textwidth}{0.5pt}\\
     \vskip-1.8ex\hfill\rule{.7\textwidth}{3pt}]


%%% macro para destacar a(s) primeira(s) letra do texto
%%% � necess�rio ter a fonte courier instalada
%
%  um texto do tipo:
%
%  \begin{document}
%    \versal{IN} THE beginning God created the heaven and the earth.  Now the
%    earth was unformed and void, and darkness was upon the face of the
%    deep; and the spirit of God hovered over the face of the waters.
%  \end{document}
%
%  ir� produzir algo como:
%
%  I I\  I THE beginning God  created the heaven and  the earth.
%  I I \ I Now the earth was unformed and void, and darkness was
%  I I  \I upon the face of the deep; and the spirit of God hov-
%  ered over the face of the waters.

%\font\largefont= pzcmi scaled 6500

\newcommand{\versal}[1]{{\noindent
    \setbox0\hbox{\largefont #1 }%
    \count0=\ht0                   % height of versal
    \count1=\baselineskip          % baselineskip
    \divide\count0 by \count1      % versal height/baselineskip
    \dimen1 = \count0\baselineskip % distance to drop versal
    \advance\count0 by 1\relax     % no of indented lines
    \dimen0=\wd0                   % width of versal
    \global\hangindent\dimen0      % set indentation distance
    \global\hangafter-\count0      % set no of indented lines
    \hskip-\dimen0\setbox0\hbox to\dimen0{\raise-\dimen1\box0\hss}%
    \dp0=0in\ht0=0in\box0}}


\renewcommand{\reftextfacebefore}{}
\renewcommand{\reftextfaceafter}{}



\newcommand{\x}{\textbf{x}\xspace}
\newcommand{\mx}{\hbox{\textbf{x}}\xspace}
\newcommand{\smx}{\hbox{{\scriptsize \textbf{x}}}\xspace}

\newcommand{\mw}{\hbox{\textbf{w}}\xspace}


\newcommand{\mpos}{\hbox{pos}\xspace}
\newcommand{\mneg}{\hbox{neg}\xspace}

\newcommand{\smpos}{\hbox{{\scriptsize pos}}\xspace}
\newcommand{\smneg}{\hbox{{\scriptsize neg}}\xspace}


% Algoritmos 
\newcommand{\coem}{\textsc{co-em}\xspace}
\newcommand{\selftraining}{\textsc{self-training}\xspace}
\newcommand{\cotesting}{\textsc{co-testing}\xspace}
\newcommand{\coemt}{\textsc{co-emt}\xspace}
\newcommand{\tsvm}{\textsc{tsvm}\xspace}
\newcommand{\coemsvl}{\textsc{co-em svl}\xspace}
\newcommand{\tritraining}{\textsc{tri-training}\xspace}
\newcommand{\cotraining}{\textsc{co-training}\xspace}




% Backref com coloca��o de "Citado na p�gina..."
\renewcommand*{\backref}[1]{}
\renewcommand*{\backrefalt}[4]{
    \ifcase #1
        N�o citado no texto.
    \or
        Citado na p�gina~#2.
    \else
        Citado nas p�ginas #2.
    \fi
}

\renewcommand{\backreftwosep}{ e~}
\renewcommand{\backreflastsep}{, e~}


\hyphenation{su-per-vi-sio-na-dos}
\hyphenation{su-per-vi-sio-na-do}


\newcommand{\gnuplot}{\textbf{GnuPlot$^{TM}$}\xspace}
\newcommand{\tfidf}{\textit{tfidf}\xspace}
\newcommand{\tflinear}{\textit{tflinear}\xspace}




\newcommand{\smooth}{\emph{smooth}\xspace}

\newcommand{\stempl}{\textbf{stem.pl}\xspace}
\newcommand{\reportpl}{\textbf{report.pl}\xspace}
\newcommand{\predictpl}{\textbf{predict.pl}\xspace}
\newcommand{\stembase}{\textbf{stembase}\xspace}
\newcommand{\textbase}{\textbf{textbase}\xspace}
\newcommand{\stoplist}{\textbf{stoplist}\xspace}


\newcommand{\sone}{\texttt{\%one\%}\xspace}
\newcommand{\stwo}{\texttt{\%two\%}\xspace}
\newcommand{\sthree}{\texttt{\%three\%}\xspace}
\newcommand{\sglobal}{\texttt{\%global\%}\xspace}
\newcommand{\stm}{\textit{stemming}\xspace}
\newcommand{\stemming}{\textit{stemming}\xspace}
\newcommand{\gram}{\textit{gram}\xspace}
\newcommand{\grams}{\textit{grams}\xspace}

\newcommand{\stems}{\textit{stems}\xspace}
\newcommand{\stem}{\textit{stem}\xspace}
\newcommand{\pretext}{\textsc{PreTexT}\xspace}
\newcommand{\script}{\textit{script}\xspace}
\newcommand{\scripts}{\textit{scripts}\xspace}
\newcommand{\stoplists}{\textit{stoplists}\xspace}
\newcommand{\stopfile}{\textit{stopfile}\xspace}
\newcommand{\stopfiles}{\textit{stopfiles}\xspace}
\newcommand{\stopword}{\textit{stopword}\xspace}
\newcommand{\stopwords}{\textit{stopwords}\xspace}


% experimentos
\newcommand{\news}{{\sc news}\xspace}
\newcommand{\course}{{\sc course}\xspace}
\newcommand{\lnai}{{\sc lnai}\xspace}
\newcommand{\ccourse}{{\tt course}\xspace}
\newcommand{\cncourse}{{\tt non-course}\xspace}
\newcommand{\sci}{{\tt sci}\xspace}
\newcommand{\talk}{{\tt talk}\xspace}
\newcommand{\texto}{\textsc{texto}\xspace}
\newcommand{\links}{\textsc{links}\xspace}

\newcommand{\cbr}{\textsc{cbr}\xspace}
\newcommand{\ilp}{\textsc{ilp}\xspace}

\newcommand{\svm}{\textsc{svm}\xspace}




\newcommand{\kmeanski}
  {$k$-{\it means}$_{ki}$\xspace}

\newcommand{\kmeans}
  {$k$-me\-ans\xspace}


\newcommand{\clustering}
  {\textit{clustering}\xspace}



\newcommand{\kparticoes}
  {$k$-parti��es\xspace}

\newcommand{\coboosting}
  {\textsc{CO-BOOSTING}\xspace}

\newcommand{\boosting}
  {{\em Boosting}\xspace}


\newcommand{\bagging}
  {{\em Bagging}\xspace}

\newcommand{\naivebayes}
  {\emph{Naive Bayes}\xspace}

\newcommand{\bayes}
  {{\em Bayes}\xspace}

\newcommand{\framework}
  {\textsc{framework}\xspace}


% bullet
\newcommand{\bb}
  {\ensuremath{\bullet}}

% neck
\newcommand{\neck}
  {{\texttt{\bf\ :-\ }}}

% circ
\newcommand{\cc}
  {\ensuremath{\circ}}

% diamond
\newcommand{\dd}
  {\ensuremath{\diamond}}

% MLC++
\newcommand{\mlc}
  {\ensuremath{\mathcal{MLC\hspace{-.05em}\raisebox{.4ex}{\tiny\bf ++}}}\xspace}

% MySQL
\newcommand{\mysql}
%  {{\sc \smaller MySQL}\xspace}
  {\ensuremath{\mathcal{M}{\sf y}\mathcal{SQL}}\xspace}

% C++
\newcommand{\cplusplus}
  {\ensuremath{\mathcal{C\hspace{-.05em}\raisebox{.4ex}{\tiny\bf ++}}}\xspace}

% C++
\newcommand{\cpp}
  {\cplusplus}

% CI
\newcommand{\ci}
  {\ensuremath{\mathcal{CI}}\xspace}
%  {{\sc ci}\xspace}

% ID3
\newcommand{\idtree}
%  {{\sc id\relsize{-2}3}\xspace}
  {\ensuremath{\mathcal{ID}3}\xspace}

% GID3*
\newcommand{\gidtree}
  {{\sc gid\relsize{-2}3*}\xspace}

% Skicat
\newcommand{\skicat}
  {{\sc \smaller Skicat}\xspace}

% PBM
%\newcommand{\pbm}
 % {{\sc \smaller pbm}\xspace}

%PBM
\newcommand{\pbm}
   {\ensuremath{\mathcal{PBM}}\xspace}

% C4.5
\newcommand{\cfourfive}
%  {{\sc c\relsize{-2}4.5}\xspace}
  {\ensuremath{\mathcal{C}4.5}\xspace}

% C4.5rules
\newcommand{\cfourfiverules}
%  {{\sc c{\relsize{-2}4.5}rules}\xspace}
  {\ensuremath{\mathcal{C}4.5{\sf rules}}\xspace}

% C4.5-rules
\newcommand{\cfourfiver}
  {{\sc c{\relsize{-2}4.5}rules}\xspace}
%  {\ensuremath{\mathcal{C}4.5{\sf rules}}\xspace}

% C4.5r
\newcommand{\cfourfiverr}
  {{\sc c{\relsize{-2}4.5}r}\xspace}
%  {\ensuremath{\mathcal{C}4.5{\sf r}}\xspace}

% C5.0
\newcommand{\cfive}
%  {{\sc c\relsize{-2}5.0}\xspace}
  {\ensuremath{\mathcal{C}5.0}\xspace}

% See5
%\newcommand{\seefive}
%  {{\sc see\relsize{-2}5}\xspace}
%  {\ensuremath{\mathcal{S}}{\sf ee5}\xspace}

% See5
\newcommand{\seefive}
  {\ensuremath{\mathcal{S}}{\sf ee5}\xspace}


% CN2
\newcommand{\cntwo}
 % {{\sc cn\relsize{-2}2}\xspace}
  {\ensuremath{\mathcal{CN}2}\xspace}

% OC1
\newcommand{\ocone}
%  {{\sc oc\relsize{-2}1}\xspace}
  {\ensuremath{\mathcal{OC}1}\xspace}

% T2
%\newcommand{\ttwo}
 % {{\sc t\relsize{-2}2}\xspace}

% MC4
%\newcommand{\mcfour}
 % {{\sc mc\relsize{-2}4}\xspace}

%MC4
\newcommand{\mcfour}
  {\ensuremath{\mathcal{MC}4}\xspace}

%T2
\newcommand{\ttwo}
  {\ensuremath{\mathcal{T}2}\xspace}

% RIPPER
\newcommand{\ripper}
%  {{\sc ripper}\xspace}
  {\ensuremath{\mathcal{R}}{\sc \em ipper}\xspace}

% Progol
\newcommand{\progol}
  {{\sc progol}\xspace}

% Naive Bayes
\newcommand{\nb}
  {{\sc nb}\xspace}

% Instance Based
\newcommand{\ib}
  {{\sc ib}\xspace}

% Labic
\newcommand{\labic}
  {{\sc labic}\xspace}

% Ruler
\newcommand{\ruler}
  {{\sc ruler}\xspace}

% Xruler
\newcommand{\xruler}
  {{\sc xruler}\xspace}

% Discover
\newcommand{\discover}
  {{\sc discover}\xspace}

% perl
\newcommand{\perl}
  {{\sc perl}\xspace}


% O-BTree
\newcommand{\obtree}
  {{\sc o-btree}\xspace}

% Focas
\newcommand{\focas}
  {{\sc focas}\xspace}

% Focus
\newcommand{\focus}
  {{\sc focus}\xspace}

% Relief
\newcommand{\relief}
  {{\sc relief}\xspace}


% Foil
\newcommand{\foil}
  {{\sc foil}\xspace}

% CART
\newcommand{\cart}
  {{\sc cart}\xspace}

% BibTeX
\def\BibTeX{{\rm B\kern-.05em{\sc i\kern-.025em b}\kern-.08em
    T\kern-.1667em\lower.7ex\hbox{E}\kern-.125emX}\xspace}
% BibTeX
\def\bibtex{\BibTeX}

% BibView
\def\BibView{{\rm B\kern-.05em{\sc i\kern-.025em b}\kern-.08em
    V\kern-.1667em\hbox{\sc iew}}\xspace}
\def\bibview{\BibView}

% Nro
\newcommand{\nro}
  {\scriptsize $^{\b{o}}$\normalsize\xspace}

% Nra
\newcommand{\nra}
  {\scriptsize $^{\b{a}}$\normalsize\xspace}

% e.g.
\newcommand{\eg}
  {{\em e.g.}\/\xspace}

% i.e.
\newcommand{\ie}
  {{\em i.e.}\/\xspace}

% trademark
\newcommand{\TM}
  {\footnotesize\ensuremath{^{\rm TM}}\normalsize\xspace}
\newcommand{\tm}
  {\TM}

%\newcommand{\IF}
%  {{\bf IF}\xspace}
\newcommand{\THEN}
  {{\bf THEN}\xspace}
% and
\newcommand{\AND}
  {{\bf AND}\xspace}
% proc
\newcommand{\PROC}
  {{\bf procedure}\xspace}
% return
\newcommand{\RETURN}
  {{\bf return}\xspace}
% in
\newcommand{\IN}
  {{\bf in}\xspace}
% in
\newcommand{\REM}[1]
  {\hspace*{1cm}{\em // #1}\xspace}


% Palavras em ingles
%\newcommand{\engl}[1]
%  {{\em#1}}
%  {\selectlanguage{english}{\em#1}\selectlanguage{brazil}}

% URL
%\newcommand{\url}[1]
%  {{\tt #1}\xspace}

\newcommand{\ip}[2]
  {(#1, #2)}

\newcommand{\seq}[3][X,1,n]
  {\lbrace #1_{#2},\ldots,\,#1_{#3} \rbrace}

\newcommand{\cartesiano}[3][X,1,n]
  {#1_{#2} \times \ldots \times #1_{#3}}

% entrada de indice simples
\newcommand{\idxa}[1]
  {#1\index{#1}}

% entrada de indice dupla
\newcommand{\idxb}[2]
  {#1 #2\index{#1!#2}}

% entrada de indice tripla
\newcommand{\idxc}[3]
  {#1 #2 #3\index{#1!#2!#3}}

% PCTeX
% figura x-size y-size filename extension label caption
%          1      2       3        4         5      6

% PCTeX
% figura x-size y-size filename extension label caption
%          1      2       3        4         5      6
\newcommand{\figura}[6]
{\begin{figure}[hbt]
   \vspace*{0.5cm}
   \setlength{\unitlength}{1.0cm}
   \centering
   \begin{picture}(#1, #2)(0, 0)
     \special{#4:./#3.#4 x=#1cm y=#2cm}
   \end{picture}
   \caption{#6}
   \label{#5}
 \end{figure}
}

%%% inclus�o de figuras
% \figurajpg{escala}{nome.extensao}{label}{caption}
\newcommand{\figurajpg}[5]
{\begin{figure}[!h]
   \setlength{\unitlength}{1.0cm}
   \centering
     \includegraphics[scale=#1]{#2}
   \caption{#3}
   \label{#4}
 \end{figure}
}

\newcommand{\figurajpgg}[5]
{\begin{figure}[!hbt]
   \setlength{\unitlength}{1.0cm}
   \centering
     \includegraphics[scale=#1]{#2}
   \caption[#5]{#4}
   \label{#3}
 \end{figure}
}

%\newcommand{\figurac}[7]
%{\begin{figure}[hbt]
 %  \vspace*{0.5cm}
 %  \setlength{\unitlength}{1.0cm}
 %  \centering
 %  \begin{picture}(#1, #2)(0, 0)
 %    \special{#4:./#3.#4 x=#1cm y=#2cm}
 %  \end{picture}
 %  \caption[#7]{#6}
 %  \label{#5}
 %\end{figure}
%}


\newcommand{\figuraa}[6]
{\begin{figure}
%   \vspace*{1cm}
   \setlength{\unitlength}{1.0cm}
   \centering
   \begin{picture}(#1, #2)(0, 0)
     \special{#4:./#3.#4 x=#1cm y=#2cm}
   \end{picture}
   \caption{#6}
   \label{#5}
 \end{figure}
}

\newcommand{\figurah}[6]
{\begin{figure}[htb]
%   \vspace*{0.2cm}
   \setlength{\unitlength}{1.0cm}
   \centering
   \begin{picture}(#1, #2)(0, 0)
     \special{#4:./#3.#4 x=#1cm y=#2cm}
   \end{picture}
   \caption{#6}
   \label{#5}
 \end{figure}
}

\newcommand{\figuraH}[6]
{\begin{figure}[H]
%   \vspace*{0.2cm}
   \setlength{\unitlength}{1.0cm}
   \centering
   \begin{picture}(#1, #2)(0, 0)
     \special{#4:./#3.#4 x=#1cm y=#2cm}
   \end{picture}
   \caption{#6}
   \label{#5}
 \end{figure}
}

\newcommand{\figurat}[6]
{\begin{figure}[tbh]
%   \vspace*{1cm}
   \setlength{\unitlength}{1.0cm}
   \centering
   \begin{picture}(#1, #2)(0, 0)
     \special{#4:./#3.#4 x=#1cm y=#2cm}
   \end{picture}
   \caption{#6}
   \label{#5}
 \end{figure}
}

\newcommand{\figurab}[6]
{\begin{figure}[bth]
%   \vspace*{0.2cm}
   \setlength{\unitlength}{1.0cm}
   \centering
   \begin{picture}(#1, #2)(0, 0)
     \special{#4:./#3.#4 x=#1cm y=#2cm}
   \end{picture}
   \caption{#6}
   \label{#5}
 \end{figure}
}

\newcommand{\figurac}[7]
{\begin{figure}[hbt]
   \vspace*{0.5cm}
   \setlength{\unitlength}{1.0cm}
   \centering
   \begin{picture}(#1, #2)(0, 0)
     \special{#4:./#3.#4 x=#1cm y=#2cm}
   \end{picture}
   \caption[#7]{#6}
   \label{#5}
 \end{figure}
}

\newcommand{\myquotation}[2]{%
  %\vspace{0.5ex}%
  {\singlespacing
    \footnotesize%
    \begin{flushright}%
      \begin{minipage}{.5\textwidth}%
        {\sf \noindent \textcolor{RawSienna}{#1}}\\
      \end{minipage}\\
      \textit{\noindent \textcolor{RawSienna}{#2}}%
    \end{flushright}}%
  %\vspace{0.5ex}
  }

%\myquotation{``Smoking kills. If you're killed, you've lost a very important part of your life.''}{Brooke Shields.}



% < x >
\newcommand{\braces}[1]
  {$<$#1$>$\xspace}

% Palavras em ingles
\newcommand{\engl}[1]
   {\emph{#1}}

% | - ou
\newcommand{\ou}
   {$|$\xspace}

% e-mail
%\newcommand{\email}
%   {\begingroup \urlstyle{tt}\Url}


\newcommand{\inducer}[1]
  {\ensuremath{\mathcal{I\hspace{-.05em}}^{#1}}\xspace}

\newcommand{\dataset}[1]
  {\ensuremath{\mathcal{D\hspace{-.05em}}^{#1}}\xspace}

\newcommand{\classifier}[1]
  {\ensuremath{\mathcal{C\hspace{-.05em}}^{#1}}\xspace}


\newcommand{\bupa}
  {{\sf \em bupa}\xspace}

%    bupa
%
%    pima
%
%cmc
%
%breast-cancer
%
%hungaria
%
%smoke
%
%crx letter
%
%hepatitis
%
%  anneal
%
%sonar
%
% genetics
%
%   dna


\begin{document}

\pagestyle{empty}
\pagenumbering{roman}

%%% insere a capa
%%% edite os campos para t�tulo, data, orientador e nome

\newcommand{\titulo}{Monitoramento de Ambientes Internos Utilizando Rob�s M�veis}


\begin{titlepage}


\ \vfill

\begin{center}
\begin{minipage}[c]{12cm}
\begin{center}
\hrulefill\\
\vspace{.5cm} {\Large \titulo}\\
\vspace{1.3cm}
\textbf{\it Heitor Luis Polidoro}\\
\vspace{.5cm}
\hrulefill\\
\end{center}
\end{minipage}
\end{center}

\vfill

\cleardoublepage


\begin{flushright}
\begin{Sbox}
\begin{minipage}{8.5cm}
\footnotesize
SERVI�O DE  P�S-GRADUA��O DO ICMC-USP\\
\\
Data de Dep�sito: \\
\\
Assinatura:\hrulefill
\end{minipage}
\end{Sbox}
\fbox{\TheSbox}
\end{flushright}


\vspace*{2cm}
\begin{center}
{\huge\sf \titulo}\footnote{Trabalho Realizado com
Aux�lio da CAPES}


\vspace*{2cm}

{\it Heitor Luis Polidoro}

\vspace*{2cm}


{\bf Orientador:}  {\it Prof. Dr. Denis Fernando Wolf}

\end{center}

\vspace*{4cm}

\begin{flushright}
\begin{minipage}{10cm}
Monografia apresentada ao Instituto de Ci�ncias Matem�ticas e de
Computa��o - ICMC-USP, para o Exame de Qualifica��o, como parte dos
requisitos necess�rios � obten��o do t�tulo de Mestre em Ci�ncias de
Computa��o e Matem�tica Computacional.
\end{minipage}
\end{flushright}

\vspace*{2cm}
\begin{center}
\textbf{USP - S�o Carlos \\fevereiro/2009}
\end{center}
\cleardoublepage



\end{titlepage}


\pagestyle{plain}


\onehalfspacing

%%% insere o resumo

%==========================================================
% SCE=567 Projeto Supervisionado ou de Graduacao II
% 
% Resumo.
% 
% Autor
% Heitor Luis Polidoro
% 
% Orientador
% Prof. Dr. Denis Fernando Wolf
%==========================================================

\pagestyle{abstractst}

\chapter*{Resumo}
\label{resumo}
A rob�tica m�vel � uma �rea de pesquisa que est� obtendo grande aten��o da comunidade cient�fica. O desenvolvimento de rob�s m�veis aut�nomos, que sejam capazes de interagir com o ambiente, aprender e de tomar decis�es corretas para que suas tarefas sejam executadas com �xito � o maior desafio em rob�tica m�vel. O desenvolvimento destes sistemas inteligentes e aut�nomos consiste em uma �rea de pesquisa multidisciplinar, considerada recente e extremamente promissora que envolve: intelig�ncia artificial, aprendizado de m�quina, estima��o estat�stica e sistemas embarcados, por exemplo. Um dos problemas fundamentais na �rea de rob�tica � a navega��o, que consiste em determinar o local para onde o rob� deve se mover, o melhor caminho at� esse local e como chegar at� esse local de forma segura, evitando obst�culos. Esse projeto de pesquisa prop�e a compara��o de algoritmos de monitoramento de ambientes internos, no qual o rob� deve percorrer o ambiente de maneira eficiente, levando em considera��o a prioridade de cada local a ser visitado e o tempo decorrido desde a �ltima visita a cada regi�o do ambiente.

\cleardoublepage

\tableofcontents
\addcontentsline{toc}{section}{Sum�rio}
\cleardoublepage

\listoffigures
\addcontentsline{toc}{section}{Lista de Figuras}
\cleardoublepage

\listoftables
\addcontentsline{toc}{section}{Lista de Tabelas}
\cleardoublepage


%\input{acron}
%\addcontentsline{toc}{section}{Lista de Abreviaturas}
%\cleardoublepage

\listofalgorithms
\addcontentsline{toc}{section}{Lista de Algoritmos}
\cleardoublepage




%%% inserir resumo
%%==========================================================
% SCE=567 Projeto Supervisionado ou de Graduacao II
% 
% Resumo.
% 
% Autor
% Heitor Luis Polidoro
% 
% Orientador
% Prof. Dr. Denis Fernando Wolf
%==========================================================

\pagestyle{abstractst}

\chapter*{Resumo}
\label{resumo}
A rob�tica m�vel � uma �rea de pesquisa que est� obtendo grande aten��o da comunidade cient�fica. O desenvolvimento de rob�s m�veis aut�nomos, que sejam capazes de interagir com o ambiente, aprender e de tomar decis�es corretas para que suas tarefas sejam executadas com �xito � o maior desafio em rob�tica m�vel. O desenvolvimento destes sistemas inteligentes e aut�nomos consiste em uma �rea de pesquisa multidisciplinar, considerada recente e extremamente promissora que envolve: intelig�ncia artificial, aprendizado de m�quina, estima��o estat�stica e sistemas embarcados, por exemplo. Um dos problemas fundamentais na �rea de rob�tica � a navega��o, que consiste em determinar o local para onde o rob� deve se mover, o melhor caminho at� esse local e como chegar at� esse local de forma segura, evitando obst�culos. Esse projeto de pesquisa prop�e a compara��o de algoritmos de monitoramento de ambientes internos, no qual o rob� deve percorrer o ambiente de maneira eficiente, levando em considera��o a prioridade de cada local a ser visitado e o tempo decorrido desde a �ltima visita a cada regi�o do ambiente.

%
%\newpage

%%% corpo do texto


\pagenumbering{arabic}

%%% inclua os seu arquivos a partir daqui

%==========================================================
% Capitulo 1: Introducao.
% 
% Autor
% Heitor Luis Polidoro
% 
% Orientador
% Prof. Dr. Denis Fernando Wolf
%==========================================================

\pagestyle{chapterst}

\chapter{Introdu��o}
\label{introducao}

\section{Contextualiza��o}
\label{contextualizacao}
A rob�tica consiste em uma �rea multidisciplinar de pesquisa, envolvendo desde elementos de engenharia mec�nica, el�trica, computa��o at� �reas de humanas como psicologia e estudos do comportamento de animais.

Inicialmente, os rob�s foram utilizados para automa��o industrial, presos em posi��es espec�ficas na linha de montagem, os bra�os rob�ticos podem se mover a uma grande velocidade e precis�o para realizar tarefas repetitivas. Gra�as a esses atributos, hoje  
mas com a evolu��o tecnol�gica os rob�s passaram a ser utilizados em outras �reas como: medicina de precis�o, ambientes perigosos, ambientes insalubres, na �rea de entretenimento, servi�os dom�sticos, etc. Nesse contexto surgiram as pesquisas para o desenvolvimento de rob�s m�veis aut�nomos, que sejam capazes de atuar em ambientes reais e reagir a situa��es desconhecidas de forma inteligente \cite{Thrun2002}.

A comunidade cient�fica aposta que sistemas rob�ticos estejam cada vez mais presentes em nossa vida cotidiana, o que torna esta �rea de pesquisa extremamente promissora e desafiadora. Segundo \cite{Gates2007}, estamos entrando numa nova era da computa��o, comparando os rob�s industriais com os mainframes de antigamente, e prev� que existir� um rob� em cada casa no futuro.

Entre todas essas diversas aplica��es da rob�tica m�vel, pode-se citar o rob� Sojourner (Figura \ref{fig:exemplo_robos}a) da NASA, que explorou e enviou fotos e outras muitas informa��es do planeta Marte para a Terra \cite{NASA}, e o rob� desenvolvido pela universidade Camegie Mellon, chamado Groundhog, que explora minas abandonadas, que al�m do risco de desabamento, em muitos casos tamb�m cont�m gases t�xicos \cite{Thrun2004}.

A rob�tica m�vel, al�m de ser uma �rea de grande potencial cient�fico, empresas de tecnologia est�o cada vez mais investindo em produtos. Como por exemplo o desenvolvimento de rob�s que realizam trabalhos dom�sticos autonomamente,  entre os quais o aspirador de p� Roomba da IRobot \cite{IRobot} (Figura \ref{fig:exemplo_robos}b) e o rob� cortador de grama Robomow (Figura \ref{fig:exemplo_robos}c) da Friendly Robotics \cite{FriendlyRobotics}. Ambos apresentam sucesso comercial. 

\figura{../imagens/exemplo_robos.jpg}{0.4}{Exemplos de rob�s.}{fig:exemplo_robos}

---------------------------------------------------------------------
MONOGRAFIA DIOGO
Neste projeto, a rob�tica � vista de uma perspectiva da ci�ncia de computa��o. Nesse contexto, o maior desafio no desenvolvimento de rob�s m�veis � a capacidade de interagir com o ambiente e tomar decis�es corretas para que suas tarefas sejam executadas com �xito \cite{Dudek2000}

Um dos maiores problemas dessa intera��o diz respeito � sua navega��o. Navega��o, conceitualmente, � o ato de guiar um rob� em um espa�o por um acaminho que possa ser percorrido levando o rob� de uma posi��o e orienta��o iniciais para uma posi��o e orienta��o finais.
---------------------------------------------------------------------
LIVRO
Neste projeto, a rob�tica � vista de uma perspectiva da ci�ncia de computa��o, no n�vel cognitivo do rob�. Cogni��o representa a decis�o e execu��o de tarefas para atingir um objetivo maior.

No caso de um rob� m�vel, o aspetco espec�fico de cogni��o diretamente ligado a uma mobilidade robusta � competencia de navega��o

---------------------------------------------------------------------

A proposta desse projeto � desenvolver uma aplica��o de rob�tica m�vel para o monitoramento de ambientes internos, onde normalmente, nesses ambientes, existem regi�es cr�ticas que possuem prioridades distintas para serem monitoradas. A prioridade (urg�ncia) da visita de cada uma dessas regi�es aumenta conforme o tempo passa e essa regi�o n�o � visitada. Ao ser visitada, a urg�ncia de uma determinada regi�o �, ent�o, diminu�da. Para a solu��o do problema descrito anteriormente, sup�es-se que o rob� tenha uma descri��o do ambiente em que atua (mapa). Existem diversas aplica��es pr�ticas para esse tipo de aplica��o. Dentre elas, pode-se citar o desenvolvimento de um rob� vigia que monitora um ambiente, dando �nfase para �reas cr�ticas. Outro exemplo consiste em um rob� que monitora a temperatura de ambientes onde esse fator � cr�tico.

Na literatura, um problema semelhante que vem sendo estudando h� muitos anos � o Problema do caixeiro-viajante da classe de problemas de roteamento de n�s, onde o caixeiro deve, dado um conjunto de cidades, partir de uma cidade base, visitar as outras somente uma vez e retornar � base \cite{Arenales2007}. 

\section{Motiva��o}
\label{motivacao}
A motiva��o deste projeto � encontrar uma estrat�gia e/ou um algoritmo eficiente para o monitoramento de ambientes internos, com a finalidade de ser utilizado em sistemas de seguran�a inteligentes, hospitais, ou qualquer outra situa��o em que o monitoramento de ambientes internos possa ser �til.

%A motiva��o deste projeto est� em expandir os conhecimentos nas �reas de navega��o e explora��o de ambientes internos, com utiliza��o de t�cnicas de intelig�ncia artificial, o interesse particular por rob�tica, o contato anterior com o rob� que ser� utilizado para os testes reais, a familiaridade com a ferramenta de desenvolvimento e com a ferramenta de simula��o Player/Stage \cite{Gerkey2003} e, tamb�m, a possibilidade de dar continuidade ao projeto iniciado na disciplina \textit{Projeto Supervisionado ou de Gradua��o I} e \textit{II}, colocando em pr�tica conceitos aprendidos nas disciplinas cursadas na gradua��o.

\section{Objetivo}
\label{objetivo}
Esse projeto tem como objetivo o desenvolvimento e compara��o entre estrat�gias e algoritmos para determinar uma seq��ncia de �reas a serem visitadas em ambientes internos com a finalidade de monitoramento desses ambientes utilizando um rob� m�vel. O problema a ser resolvido consiste na divis�o de um ambiente previamente conhecido em �reas de interesse. A cada uma dessas �reas � atribu�do um valor (peso) referente a sua import�ncia de monitoramento. A prioridade com que o rob� deve visitar determinadas �reas � calculada com base na import�ncia dessas �reas e no tempo decorrido desde a sua �ltima visita. �reas de maior import�ncia devem ser visitadas mais freq�entemente. 

As avalia��es consistem na compara��o dos algoritmos e estrat�gias. Ser�o utilizados dois crit�rios de compara��o, um crit�rio comparando a freq��ncia relativa de cada sala com sua prioridade relativa, onde o melhor resultado � aquele em que a freq��ncia relativa se aproximar mais da prioridade relativa. O segundo crit�rio � um gr�fico mostrando a progress�o da somat�ria dos graus de urg�ncia de todas as salas.

%\section{Organiza��o da Monografia}
%\label{organizacao_monografia}
%Organiza��o da monografia.

\chapter{Minera��o de Textos}
\label{cap:textmining}

\section{Etapas da Minera��o de Textos}

\subsection{Pr�-processamento}

\subsection{Extra��o de Padr�es}

\subsection{Outras Etapas}



\section{Tarefas de Minera��o de Textos}

\subsection{Classifica��o}

\subsection{Outras Tarefas}
\chapter{Aprendizado Semi-supervisionado Multivis�o em Textos}
\label{cap:semi}

\section{Aprendizado Semi-supervisionado}

\section{O Algoritmo \cotraining}

\section{Extra��o de Vis�es em Textos}
\chapter{Extra��o de Terminologia}
\label{cap:term}

\chapter{Ferramentas de Apoio}
\label{cap:tools}

\section{\pretext -- um Pr�-processador de Textos}

\section{O Ambiente para Aprendizado \discover}

\section{Ferramentas para Extra��o de Terminologia}
\chapter{Plano de Trabalho}
\label{cap:plano}

\section{Metodologia}



\section{Avalia��o}

\subsection{\textit{Corpora} Dispon�veis}

\subsection{Experimentos}



\section{Requisitos, Atividades e Cronograma}

\subsection{Requisitos do Programa}

\subsection{Atividades Futuras e Cronograma}

\chapter{Resultados Esperados}
\label{cap:conclusao}


\appendix
\bibliographystyle{apalike}

%%% arquivo .bib
\bibliography{main}
\addcontentsline{toc}{chapter}{Refer�ncias}


% Set the ending of a LaTeX document
\end{document}
