
\documentclass[12pt,twoside]{book}

\usepackage[brazilian]{babel}
\usepackage[T1]{fontenc} 
\usepackage{ae}
\usepackage[latin1]{inputenc}
%\usepackage[portuges,brazilian]{babel}
\usepackage{subfigure}
\usepackage{natbib}
\usepackage{latexsym}
\usepackage{setspace}
\usepackage{xspace}
\usepackage{acronym}
\usepackage[brazil]{varioref}                 %%% referencias com p�gina \vref
\usepackage{indentfirst}                      %%% identa primeiro par�grafo
\usepackage{subfigure}                        %%% figuras um ao lado do outro
\usepackage[sf,sl,outermarks]{titlesec}       %%% definir t�tulos de se��o
\usepackage{graphicx}


%%% boxes
\usepackage{fancybox}
\usepackage{fancyvrb}
\usepackage{color}

%%% dimens�es do documento
\usepackage[pdftex]{geometry}
  \geometry{a4paper,left=3cm,right=2cm,top=2.0cm,bottom=2cm,twoside}
%%% cria links no arquivo .pdf
%%% comentar as linhas na vers�o para impressao
%\usepackage[pdftex,pdfpagelabels,pagebackref]{hyperref}
\usepackage{hyperref}
\hypersetup{colorlinks=true,linkcolor=black,citecolor=black,hypertexnames=false}

%%fonte
\usepackage{bookman}
\usepackage[T1]{fontenc}


%%para os numeros reais
\usepackage{amsfonts}

\DeclareFixedFont{\numberfont}{T1}{phv}{bx}{n}{2cm}


%%%
%%% Algoritmos, for human beings
%%%

\usepackage[chapter]{algorithm}
\usepackage{algorithmic}

\renewcommand{\algorithmicend}{\textbf{fim}}
\renewcommand{\algorithmicif}{\textbf{se}}
\renewcommand{\algorithmicthen}{\textbf{ent�o}}
\renewcommand{\algorithmicelse}{\textbf{sen�o}}
\renewcommand{\algorithmicfor}{\textbf{para cada}}
\renewcommand{\algorithmicforall}{\textbf{para todos}}
\renewcommand{\algorithmicdo}{\textbf{fa�a}}
\renewcommand{\algorithmicwhile}{\textbf{enquanto}}
\renewcommand{\algorithmicrepeat}{\textbf{repetir}}
\renewcommand{\algorithmicuntil}{\textbf{at�}}
\floatname{algorithm}{Algoritmo}

%%%
%%% redefine o formato do t�tulo
%%%

\titleformat{\chapter}[display]
  {\normalfont\Large\sffamily
  }
  {%\titlerule[3pt]%
   \filright
   \rule[32pt]{.7\linewidth}{4pt}
   \hspace{-11pt}
   \shadowbox{
   \begin{minipage}{.18\linewidth}
     \begin{center}
       \textsc{\Large\chaptertitlename}\\
       \vspace{1ex}
       {\numberfont\color[gray]{0.5} \thechapter}\\
       \vspace{1ex}
     \end{center}
   \end{minipage}}
  }
  {0pt}
  {\filcenter
   \Huge
   }
  [\hfill\rule{.8\textwidth}{0.5pt}\\
     \vskip-1.8ex\hfill\rule{.7\textwidth}{3pt}]


%%% macro para destacar a(s) primeira(s) letra do texto
%%% � necess�rio ter a fonte courier instalada
%
%  um texto do tipo:
%
%  \begin{document}
%    \versal{IN} THE beginning God created the heaven and the earth.  Now the
%    earth was unformed and void, and darkness was upon the face of the
%    deep; and the spirit of God hovered over the face of the waters.
%  \end{document}
%
%  ir� produzir algo como:
%
%  I I\  I THE beginning God  created the heaven and  the earth.
%  I I \ I Now the earth was unformed and void, and darkness was
%  I I  \I upon the face of the deep; and the spirit of God hov-
%  ered over the face of the waters.

%\font\largefont= pzcmi scaled 6500

\newcommand{\versal}[1]{{\noindent
    \setbox0\hbox{\largefont #1 }%
    \count0=\ht0                   % height of versal
    \count1=\baselineskip          % baselineskip
    \divide\count0 by \count1      % versal height/baselineskip
    \dimen1 = \count0\baselineskip % distance to drop versal
    \advance\count0 by 1\relax     % no of indented lines
    \dimen0=\wd0                   % width of versal
    \global\hangindent\dimen0      % set indentation distance
    \global\hangafter-\count0      % set no of indented lines
    \hskip-\dimen0\setbox0\hbox to\dimen0{\raise-\dimen1\box0\hss}%
    \dp0=0in\ht0=0in\box0}}


\renewcommand{\reftextfacebefore}{}
\renewcommand{\reftextfaceafter}{}



\newcommand{\x}{\textbf{x}\xspace}
\newcommand{\mx}{\hbox{\textbf{x}}\xspace}
\newcommand{\smx}{\hbox{{\scriptsize \textbf{x}}}\xspace}

\newcommand{\mw}{\hbox{\textbf{w}}\xspace}


\newcommand{\mpos}{\hbox{pos}\xspace}
\newcommand{\mneg}{\hbox{neg}\xspace}

\newcommand{\smpos}{\hbox{{\scriptsize pos}}\xspace}
\newcommand{\smneg}{\hbox{{\scriptsize neg}}\xspace}


% Algoritmos 
\newcommand{\coem}{\textsc{co-em}\xspace}
\newcommand{\selftraining}{\textsc{self-training}\xspace}
\newcommand{\cotesting}{\textsc{co-testing}\xspace}
\newcommand{\coemt}{\textsc{co-emt}\xspace}
\newcommand{\tsvm}{\textsc{tsvm}\xspace}
\newcommand{\coemsvl}{\textsc{co-em svl}\xspace}
\newcommand{\tritraining}{\textsc{tri-training}\xspace}
\newcommand{\cotraining}{\textsc{co-training}\xspace}




% Backref com coloca��o de "Citado na p�gina..."
\renewcommand*{\backref}[1]{}
\renewcommand*{\backrefalt}[4]{
    \ifcase #1
        N�o citado no texto.
    \or
        Citado na p�gina~#2.
    \else
        Citado nas p�ginas #2.
    \fi
}

\renewcommand{\backreftwosep}{ e~}
\renewcommand{\backreflastsep}{, e~}


\hyphenation{su-per-vi-sio-na-dos}
\hyphenation{su-per-vi-sio-na-do}


\newcommand{\gnuplot}{\textbf{GnuPlot$^{TM}$}\xspace}
\newcommand{\tfidf}{\textit{tfidf}\xspace}
\newcommand{\tflinear}{\textit{tflinear}\xspace}




\newcommand{\smooth}{\emph{smooth}\xspace}

\newcommand{\stempl}{\textbf{stem.pl}\xspace}
\newcommand{\reportpl}{\textbf{report.pl}\xspace}
\newcommand{\predictpl}{\textbf{predict.pl}\xspace}
\newcommand{\stembase}{\textbf{stembase}\xspace}
\newcommand{\textbase}{\textbf{textbase}\xspace}
\newcommand{\stoplist}{\textbf{stoplist}\xspace}


\newcommand{\sone}{\texttt{\%one\%}\xspace}
\newcommand{\stwo}{\texttt{\%two\%}\xspace}
\newcommand{\sthree}{\texttt{\%three\%}\xspace}
\newcommand{\sglobal}{\texttt{\%global\%}\xspace}
\newcommand{\stm}{\textit{stemming}\xspace}
\newcommand{\stemming}{\textit{stemming}\xspace}
\newcommand{\gram}{\textit{gram}\xspace}
\newcommand{\grams}{\textit{grams}\xspace}

\newcommand{\stems}{\textit{stems}\xspace}
\newcommand{\stem}{\textit{stem}\xspace}
\newcommand{\pretext}{\textsc{PreTexT}\xspace}
\newcommand{\script}{\textit{script}\xspace}
\newcommand{\scripts}{\textit{scripts}\xspace}
\newcommand{\stoplists}{\textit{stoplists}\xspace}
\newcommand{\stopfile}{\textit{stopfile}\xspace}
\newcommand{\stopfiles}{\textit{stopfiles}\xspace}
\newcommand{\stopword}{\textit{stopword}\xspace}
\newcommand{\stopwords}{\textit{stopwords}\xspace}


% experimentos
\newcommand{\news}{{\sc news}\xspace}
\newcommand{\course}{{\sc course}\xspace}
\newcommand{\lnai}{{\sc lnai}\xspace}
\newcommand{\ccourse}{{\tt course}\xspace}
\newcommand{\cncourse}{{\tt non-course}\xspace}
\newcommand{\sci}{{\tt sci}\xspace}
\newcommand{\talk}{{\tt talk}\xspace}
\newcommand{\texto}{\textsc{texto}\xspace}
\newcommand{\links}{\textsc{links}\xspace}

\newcommand{\cbr}{\textsc{cbr}\xspace}
\newcommand{\ilp}{\textsc{ilp}\xspace}

\newcommand{\svm}{\textsc{svm}\xspace}




\newcommand{\kmeanski}
  {$k$-{\it means}$_{ki}$\xspace}

\newcommand{\kmeans}
  {$k$-me\-ans\xspace}


\newcommand{\clustering}
  {\textit{clustering}\xspace}



\newcommand{\kparticoes}
  {$k$-parti��es\xspace}

\newcommand{\coboosting}
  {\textsc{CO-BOOSTING}\xspace}

\newcommand{\boosting}
  {{\em Boosting}\xspace}


\newcommand{\bagging}
  {{\em Bagging}\xspace}

\newcommand{\naivebayes}
  {\emph{Naive Bayes}\xspace}

\newcommand{\bayes}
  {{\em Bayes}\xspace}

\newcommand{\framework}
  {\textsc{framework}\xspace}


% bullet
\newcommand{\bb}
  {\ensuremath{\bullet}}

% neck
\newcommand{\neck}
  {{\texttt{\bf\ :-\ }}}

% circ
\newcommand{\cc}
  {\ensuremath{\circ}}

% diamond
\newcommand{\dd}
  {\ensuremath{\diamond}}

% MLC++
\newcommand{\mlc}
  {\ensuremath{\mathcal{MLC\hspace{-.05em}\raisebox{.4ex}{\tiny\bf ++}}}\xspace}

% MySQL
\newcommand{\mysql}
%  {{\sc \smaller MySQL}\xspace}
  {\ensuremath{\mathcal{M}{\sf y}\mathcal{SQL}}\xspace}

% C++
\newcommand{\cplusplus}
  {\ensuremath{\mathcal{C\hspace{-.05em}\raisebox{.4ex}{\tiny\bf ++}}}\xspace}

% C++
\newcommand{\cpp}
  {\cplusplus}

% CI
\newcommand{\ci}
  {\ensuremath{\mathcal{CI}}\xspace}
%  {{\sc ci}\xspace}

% ID3
\newcommand{\idtree}
%  {{\sc id\relsize{-2}3}\xspace}
  {\ensuremath{\mathcal{ID}3}\xspace}

% GID3*
\newcommand{\gidtree}
  {{\sc gid\relsize{-2}3*}\xspace}

% Skicat
\newcommand{\skicat}
  {{\sc \smaller Skicat}\xspace}

% PBM
%\newcommand{\pbm}
 % {{\sc \smaller pbm}\xspace}

%PBM
\newcommand{\pbm}
   {\ensuremath{\mathcal{PBM}}\xspace}

% C4.5
\newcommand{\cfourfive}
%  {{\sc c\relsize{-2}4.5}\xspace}
  {\ensuremath{\mathcal{C}4.5}\xspace}

% C4.5rules
\newcommand{\cfourfiverules}
%  {{\sc c{\relsize{-2}4.5}rules}\xspace}
  {\ensuremath{\mathcal{C}4.5{\sf rules}}\xspace}

% C4.5-rules
\newcommand{\cfourfiver}
  {{\sc c{\relsize{-2}4.5}rules}\xspace}
%  {\ensuremath{\mathcal{C}4.5{\sf rules}}\xspace}

% C4.5r
\newcommand{\cfourfiverr}
  {{\sc c{\relsize{-2}4.5}r}\xspace}
%  {\ensuremath{\mathcal{C}4.5{\sf r}}\xspace}

% C5.0
\newcommand{\cfive}
%  {{\sc c\relsize{-2}5.0}\xspace}
  {\ensuremath{\mathcal{C}5.0}\xspace}

% See5
%\newcommand{\seefive}
%  {{\sc see\relsize{-2}5}\xspace}
%  {\ensuremath{\mathcal{S}}{\sf ee5}\xspace}

% See5
\newcommand{\seefive}
  {\ensuremath{\mathcal{S}}{\sf ee5}\xspace}


% CN2
\newcommand{\cntwo}
 % {{\sc cn\relsize{-2}2}\xspace}
  {\ensuremath{\mathcal{CN}2}\xspace}

% OC1
\newcommand{\ocone}
%  {{\sc oc\relsize{-2}1}\xspace}
  {\ensuremath{\mathcal{OC}1}\xspace}

% T2
%\newcommand{\ttwo}
 % {{\sc t\relsize{-2}2}\xspace}

% MC4
%\newcommand{\mcfour}
 % {{\sc mc\relsize{-2}4}\xspace}

%MC4
\newcommand{\mcfour}
  {\ensuremath{\mathcal{MC}4}\xspace}

%T2
\newcommand{\ttwo}
  {\ensuremath{\mathcal{T}2}\xspace}

% RIPPER
\newcommand{\ripper}
%  {{\sc ripper}\xspace}
  {\ensuremath{\mathcal{R}}{\sc \em ipper}\xspace}

% Progol
\newcommand{\progol}
  {{\sc progol}\xspace}

% Naive Bayes
\newcommand{\nb}
  {{\sc nb}\xspace}

% Instance Based
\newcommand{\ib}
  {{\sc ib}\xspace}

% Labic
\newcommand{\labic}
  {{\sc labic}\xspace}

% Ruler
\newcommand{\ruler}
  {{\sc ruler}\xspace}

% Xruler
\newcommand{\xruler}
  {{\sc xruler}\xspace}

% Discover
\newcommand{\discover}
  {{\sc discover}\xspace}

% perl
\newcommand{\perl}
  {{\sc perl}\xspace}


% O-BTree
\newcommand{\obtree}
  {{\sc o-btree}\xspace}

% Focas
\newcommand{\focas}
  {{\sc focas}\xspace}

% Focus
\newcommand{\focus}
  {{\sc focus}\xspace}

% Relief
\newcommand{\relief}
  {{\sc relief}\xspace}


% Foil
\newcommand{\foil}
  {{\sc foil}\xspace}

% CART
\newcommand{\cart}
  {{\sc cart}\xspace}

% BibTeX
\def\BibTeX{{\rm B\kern-.05em{\sc i\kern-.025em b}\kern-.08em
    T\kern-.1667em\lower.7ex\hbox{E}\kern-.125emX}\xspace}
% BibTeX
\def\bibtex{\BibTeX}

% BibView
\def\BibView{{\rm B\kern-.05em{\sc i\kern-.025em b}\kern-.08em
    V\kern-.1667em\hbox{\sc iew}}\xspace}
\def\bibview{\BibView}

% Nro
\newcommand{\nro}
  {\scriptsize $^{\b{o}}$\normalsize\xspace}

% Nra
\newcommand{\nra}
  {\scriptsize $^{\b{a}}$\normalsize\xspace}

% e.g.
\newcommand{\eg}
  {{\em e.g.}\/\xspace}

% i.e.
\newcommand{\ie}
  {{\em i.e.}\/\xspace}

% trademark
\newcommand{\TM}
  {\footnotesize\ensuremath{^{\rm TM}}\normalsize\xspace}
\newcommand{\tm}
  {\TM}

%\newcommand{\IF}
%  {{\bf IF}\xspace}
\newcommand{\THEN}
  {{\bf THEN}\xspace}
% and
\newcommand{\AND}
  {{\bf AND}\xspace}
% proc
\newcommand{\PROC}
  {{\bf procedure}\xspace}
% return
\newcommand{\RETURN}
  {{\bf return}\xspace}
% in
\newcommand{\IN}
  {{\bf in}\xspace}
% in
\newcommand{\REM}[1]
  {\hspace*{1cm}{\em // #1}\xspace}


% Palavras em ingles
%\newcommand{\engl}[1]
%  {{\em#1}}
%  {\selectlanguage{english}{\em#1}\selectlanguage{brazil}}

% URL
%\newcommand{\url}[1]
%  {{\tt #1}\xspace}

\newcommand{\ip}[2]
  {(#1, #2)}

\newcommand{\seq}[3][X,1,n]
  {\lbrace #1_{#2},\ldots,\,#1_{#3} \rbrace}

\newcommand{\cartesiano}[3][X,1,n]
  {#1_{#2} \times \ldots \times #1_{#3}}

% entrada de indice simples
\newcommand{\idxa}[1]
  {#1\index{#1}}

% entrada de indice dupla
\newcommand{\idxb}[2]
  {#1 #2\index{#1!#2}}

% entrada de indice tripla
\newcommand{\idxc}[3]
  {#1 #2 #3\index{#1!#2!#3}}

% PCTeX
% figura x-size y-size filename extension label caption
%          1      2       3        4         5      6

% PCTeX
% figura x-size y-size filename extension label caption
%          1      2       3        4         5      6
\newcommand{\figura}[6]
{\begin{figure}[hbt]
   \vspace*{0.5cm}
   \setlength{\unitlength}{1.0cm}
   \centering
   \begin{picture}(#1, #2)(0, 0)
     \special{#4:./#3.#4 x=#1cm y=#2cm}
   \end{picture}
   \caption{#6}
   \label{#5}
 \end{figure}
}

%%% inclus�o de figuras
% \figurajpg{escala}{nome.extensao}{label}{caption}
\newcommand{\figurajpg}[5]
{\begin{figure}[!h]
   \setlength{\unitlength}{1.0cm}
   \centering
     \includegraphics[scale=#1]{#2}
   \caption{#3}
   \label{#4}
 \end{figure}
}

\newcommand{\figurajpgg}[5]
{\begin{figure}[!hbt]
   \setlength{\unitlength}{1.0cm}
   \centering
     \includegraphics[scale=#1]{#2}
   \caption[#5]{#4}
   \label{#3}
 \end{figure}
}

%\newcommand{\figurac}[7]
%{\begin{figure}[hbt]
 %  \vspace*{0.5cm}
 %  \setlength{\unitlength}{1.0cm}
 %  \centering
 %  \begin{picture}(#1, #2)(0, 0)
 %    \special{#4:./#3.#4 x=#1cm y=#2cm}
 %  \end{picture}
 %  \caption[#7]{#6}
 %  \label{#5}
 %\end{figure}
%}


\newcommand{\figuraa}[6]
{\begin{figure}
%   \vspace*{1cm}
   \setlength{\unitlength}{1.0cm}
   \centering
   \begin{picture}(#1, #2)(0, 0)
     \special{#4:./#3.#4 x=#1cm y=#2cm}
   \end{picture}
   \caption{#6}
   \label{#5}
 \end{figure}
}

\newcommand{\figurah}[6]
{\begin{figure}[htb]
%   \vspace*{0.2cm}
   \setlength{\unitlength}{1.0cm}
   \centering
   \begin{picture}(#1, #2)(0, 0)
     \special{#4:./#3.#4 x=#1cm y=#2cm}
   \end{picture}
   \caption{#6}
   \label{#5}
 \end{figure}
}

\newcommand{\figuraH}[6]
{\begin{figure}[H]
%   \vspace*{0.2cm}
   \setlength{\unitlength}{1.0cm}
   \centering
   \begin{picture}(#1, #2)(0, 0)
     \special{#4:./#3.#4 x=#1cm y=#2cm}
   \end{picture}
   \caption{#6}
   \label{#5}
 \end{figure}
}

\newcommand{\figurat}[6]
{\begin{figure}[tbh]
%   \vspace*{1cm}
   \setlength{\unitlength}{1.0cm}
   \centering
   \begin{picture}(#1, #2)(0, 0)
     \special{#4:./#3.#4 x=#1cm y=#2cm}
   \end{picture}
   \caption{#6}
   \label{#5}
 \end{figure}
}

\newcommand{\figurab}[6]
{\begin{figure}[bth]
%   \vspace*{0.2cm}
   \setlength{\unitlength}{1.0cm}
   \centering
   \begin{picture}(#1, #2)(0, 0)
     \special{#4:./#3.#4 x=#1cm y=#2cm}
   \end{picture}
   \caption{#6}
   \label{#5}
 \end{figure}
}

\newcommand{\figurac}[7]
{\begin{figure}[hbt]
   \vspace*{0.5cm}
   \setlength{\unitlength}{1.0cm}
   \centering
   \begin{picture}(#1, #2)(0, 0)
     \special{#4:./#3.#4 x=#1cm y=#2cm}
   \end{picture}
   \caption[#7]{#6}
   \label{#5}
 \end{figure}
}

\newcommand{\myquotation}[2]{%
  %\vspace{0.5ex}%
  {\singlespacing
    \footnotesize%
    \begin{flushright}%
      \begin{minipage}{.5\textwidth}%
        {\sf \noindent \textcolor{RawSienna}{#1}}\\
      \end{minipage}\\
      \textit{\noindent \textcolor{RawSienna}{#2}}%
    \end{flushright}}%
  %\vspace{0.5ex}
  }

%\myquotation{``Smoking kills. If you're killed, you've lost a very important part of your life.''}{Brooke Shields.}



% < x >
\newcommand{\braces}[1]
  {$<$#1$>$\xspace}

% Palavras em ingles
\newcommand{\engl}[1]
   {\emph{#1}}

% | - ou
\newcommand{\ou}
   {$|$\xspace}

% e-mail
%\newcommand{\email}
%   {\begingroup \urlstyle{tt}\Url}


\newcommand{\inducer}[1]
  {\ensuremath{\mathcal{I\hspace{-.05em}}^{#1}}\xspace}

\newcommand{\dataset}[1]
  {\ensuremath{\mathcal{D\hspace{-.05em}}^{#1}}\xspace}

\newcommand{\classifier}[1]
  {\ensuremath{\mathcal{C\hspace{-.05em}}^{#1}}\xspace}


\newcommand{\bupa}
  {{\sf \em bupa}\xspace}

%    bupa
%
%    pima
%
%cmc
%
%breast-cancer
%
%hungaria
%
%smoke
%
%crx letter
%
%hepatitis
%
%  anneal
%
%sonar
%
% genetics
%
%   dna


\begin{document}

\pagestyle{empty}
\pagenumbering{roman}

%%% insere a capa
%%% edite os campos para t�tulo, data, orientador e nome

\newcommand{\titulo}{Planejamento de trajet�ria em ambientes com prioridades din�micas}

\begin{titlepage}


\ \vfill

\begin{center}
\begin{minipage}[c]{12cm}
\begin{center}
\hrulefill\\
\vspace{.5cm} {\Large \titulo}\\
\vspace{1.3cm}
\textbf{\it Heitor Luis Polidoro}\\
\vspace{.5cm}
\hrulefill\\
\end{center}
\end{minipage}
\end{center}

\vfill

\cleardoublepage


\begin{flushright}
\begin{Sbox}
\begin{minipage}{8.5cm}
\footnotesize
SERVI�O DE  P�S-GRADUA��O DO ICMC-USP\\
\\
Data de Dep�sito: \\
\\
Assinatura:\hrulefill
\end{minipage}
\end{Sbox}
\fbox{\TheSbox}
\end{flushright}


\vspace*{2cm}
\begin{center}
{\huge\sf \titulo}\footnote{Trabalho Realizado com
Aux�lio da CAPES}


\vspace*{2cm}

{\it Heitor Luis Polidoro}

\vspace*{2cm}


{\bf Orientador:}  {\it Prof. Dr. Denis Fernando Wolf}

\end{center}

\vspace*{4cm}

\begin{flushright}
\begin{minipage}{10cm}
Monografia apresentada ao Instituto de Ci�ncias Matem�ticas e de
Computa��o - ICMC-USP, para o Exame de Qualifica��o, como parte dos
requisitos necess�rios � obten��o do t�tulo de Mestre em Ci�ncias de
Computa��o e Matem�tica Computacional.
\end{minipage}
\end{flushright}

\vspace*{2cm}
\begin{center}
\textbf{USP - S�o Carlos \\fevereiro/2009}
\end{center}
\cleardoublepage



\end{titlepage}


\pagestyle{plain}

\onehalfspacing

%%% insere a dedicatoria

%==========================================================
% Dedicat�ria.
% 
% Autor
% Heitor Luis Polidoro
% 
% Orientador
% Prof. Dr. Denis Fernando Wolf
%==========================================================

\chapter*{Dedicat�ria}
Aos meus pais, Heitor e Sonia, pelo esfor�o, carinho, educa��o, dedica��o e amor.
\cleardoublepage

%%% insere os agradecimentos

%==========================================================
% Agradecimentos.
% 
% Autor
% Heitor Luis Polidoro
% 
% Orientador
% Prof. Dr. Denis Fernando Wolf
%==========================================================

\chapter*{Agradecimentos}
Agrade�o primeiramente � Deus por me permitir realizar este projeto de Mestrado.

Ao professor Denis Fernando Wolf pela orienta��o, incentivo e paci�ncia. 

Ao Instituto de Ci�ncias Matem�ticas e de Computa��o - ICMC-USP - por prover os recursos necess�rios para minha forma��o e conclus�o desse projeto de Mestrado.

� Coordena��o de Aperfei�oamento de Pessoal de N�vel Superior - CAPES - pelo suporte financeiro � esse projeto de Mestrado.

E a todos aqueles que, de alguma forma, contribu�ram para a conclus�o desse projeto de Mestrado.

\cleardoublepage

%%% insere a citacao

%==========================================================
% Cita��o.
% 
% Autor
% Heitor Luis Polidoro
% 
% Orientador
% Prof. Dr. Denis Fernando Wolf
%==========================================================

\newpage

\begin{table}[b]
		\begin{tabular}{r}
		\\\\\\\\\\\\\\\\\\\\\\\\\\\\
		\\\\\\\\\\\\\\\\\\\\\\\\\\\\
		\\\\\\\\\\\\\\\\\\\\\\\\\\\\

``\textit{Na mudan�a de atitude n�o h� mal que n�o se mude nem doen�a sem cura.}\\
\textit{Na mudan�a de postura a gente fica mais seguro.}\\
\textit{Na mudan�a do presente a gente molda o futuro.}'' \\
\\
\textbf{Gabriel, O Pensador}	\\

\end{tabular}
\end{table}
	
\cleardoublepage

%%% insere o resumo

%==========================================================
% SCE=567 Projeto Supervisionado ou de Graduacao II
% 
% Resumo.
% 
% Autor
% Heitor Luis Polidoro
% 
% Orientador
% Prof. Dr. Denis Fernando Wolf
%==========================================================

\pagestyle{abstractst}

\chapter*{Resumo}
\label{resumo}
A rob�tica m�vel � uma �rea de pesquisa que est� obtendo grande aten��o da comunidade cient�fica. O desenvolvimento de rob�s m�veis aut�nomos, que sejam capazes de interagir com o ambiente, aprender e de tomar decis�es corretas para que suas tarefas sejam executadas com �xito � o maior desafio em rob�tica m�vel. O desenvolvimento destes sistemas inteligentes e aut�nomos consiste em uma �rea de pesquisa multidisciplinar, considerada recente e extremamente promissora que envolve: intelig�ncia artificial, aprendizado de m�quina, estima��o estat�stica e sistemas embarcados, por exemplo. Um dos problemas fundamentais na �rea de rob�tica � a navega��o, que consiste em determinar o local para onde o rob� deve se mover, o melhor caminho at� esse local e como chegar at� esse local de forma segura, evitando obst�culos. Esse projeto de pesquisa prop�e a compara��o de algoritmos de monitoramento de ambientes internos, no qual o rob� deve percorrer o ambiente de maneira eficiente, levando em considera��o a prioridade de cada local a ser visitado e o tempo decorrido desde a �ltima visita a cada regi�o do ambiente.

\cleardoublepage
\tableofcontents
\addcontentsline{toc}{section}{Sum�rio}
\cleardoublepage

\listoffigures
\addcontentsline{toc}{section}{Lista de Figuras}
\cleardoublepage

\listoftables
\addcontentsline{toc}{section}{Lista de Tabelas}
\cleardoublepage

%\input{acron}
%\addcontentsline{toc}{section}{Lista de Abreviaturas}
%\cleardoublepage

%\listofalgorithms
%\addcontentsline{toc}{section}{Lista de Algoritmos}
%\cleardoublepage

%%% inserir resumo
%%==========================================================
% SCE=567 Projeto Supervisionado ou de Graduacao II
% 
% Resumo.
% 
% Autor
% Heitor Luis Polidoro
% 
% Orientador
% Prof. Dr. Denis Fernando Wolf
%==========================================================

\pagestyle{abstractst}

\chapter*{Resumo}
\label{resumo}
A rob�tica m�vel � uma �rea de pesquisa que est� obtendo grande aten��o da comunidade cient�fica. O desenvolvimento de rob�s m�veis aut�nomos, que sejam capazes de interagir com o ambiente, aprender e de tomar decis�es corretas para que suas tarefas sejam executadas com �xito � o maior desafio em rob�tica m�vel. O desenvolvimento destes sistemas inteligentes e aut�nomos consiste em uma �rea de pesquisa multidisciplinar, considerada recente e extremamente promissora que envolve: intelig�ncia artificial, aprendizado de m�quina, estima��o estat�stica e sistemas embarcados, por exemplo. Um dos problemas fundamentais na �rea de rob�tica � a navega��o, que consiste em determinar o local para onde o rob� deve se mover, o melhor caminho at� esse local e como chegar at� esse local de forma segura, evitando obst�culos. Esse projeto de pesquisa prop�e a compara��o de algoritmos de monitoramento de ambientes internos, no qual o rob� deve percorrer o ambiente de maneira eficiente, levando em considera��o a prioridade de cada local a ser visitado e o tempo decorrido desde a �ltima visita a cada regi�o do ambiente.



%%% corpo do texto

\pagenumbering{arabic}

%%% inclua os arquivos a partir daqui

%==========================================================
% Capitulo 1: Introducao.
% 
% Autor
% Heitor Luis Polidoro
% 
% Orientador
% Prof. Dr. Denis Fernando Wolf
%==========================================================

\pagestyle{chapterst}

\chapter{Introdu��o}
\label{introducao}

\section{Contextualiza��o}
\label{contextualizacao}
A rob�tica consiste em uma �rea multidisciplinar de pesquisa, envolvendo desde elementos de engenharia mec�nica, el�trica, computa��o at� �reas de humanas como psicologia e estudos do comportamento de animais.

Inicialmente, os rob�s foram utilizados para automa��o industrial, presos em posi��es espec�ficas na linha de montagem, os bra�os rob�ticos podem se mover a uma grande velocidade e precis�o para realizar tarefas repetitivas. Gra�as a esses atributos, hoje  
mas com a evolu��o tecnol�gica os rob�s passaram a ser utilizados em outras �reas como: medicina de precis�o, ambientes perigosos, ambientes insalubres, na �rea de entretenimento, servi�os dom�sticos, etc. Nesse contexto surgiram as pesquisas para o desenvolvimento de rob�s m�veis aut�nomos, que sejam capazes de atuar em ambientes reais e reagir a situa��es desconhecidas de forma inteligente \cite{Thrun2002}.

A comunidade cient�fica aposta que sistemas rob�ticos estejam cada vez mais presentes em nossa vida cotidiana, o que torna esta �rea de pesquisa extremamente promissora e desafiadora. Segundo \cite{Gates2007}, estamos entrando numa nova era da computa��o, comparando os rob�s industriais com os mainframes de antigamente, e prev� que existir� um rob� em cada casa no futuro.

Entre todas essas diversas aplica��es da rob�tica m�vel, pode-se citar o rob� Sojourner (Figura \ref{fig:exemplo_robos}a) da NASA, que explorou e enviou fotos e outras muitas informa��es do planeta Marte para a Terra \cite{NASA}, e o rob� desenvolvido pela universidade Camegie Mellon, chamado Groundhog, que explora minas abandonadas, que al�m do risco de desabamento, em muitos casos tamb�m cont�m gases t�xicos \cite{Thrun2004}.

A rob�tica m�vel, al�m de ser uma �rea de grande potencial cient�fico, empresas de tecnologia est�o cada vez mais investindo em produtos. Como por exemplo o desenvolvimento de rob�s que realizam trabalhos dom�sticos autonomamente,  entre os quais o aspirador de p� Roomba da IRobot \cite{IRobot} (Figura \ref{fig:exemplo_robos}b) e o rob� cortador de grama Robomow (Figura \ref{fig:exemplo_robos}c) da Friendly Robotics \cite{FriendlyRobotics}. Ambos apresentam sucesso comercial. 

\figura{../imagens/exemplo_robos.jpg}{0.4}{Exemplos de rob�s.}{fig:exemplo_robos}

---------------------------------------------------------------------
MONOGRAFIA DIOGO
Neste projeto, a rob�tica � vista de uma perspectiva da ci�ncia de computa��o. Nesse contexto, o maior desafio no desenvolvimento de rob�s m�veis � a capacidade de interagir com o ambiente e tomar decis�es corretas para que suas tarefas sejam executadas com �xito \cite{Dudek2000}

Um dos maiores problemas dessa intera��o diz respeito � sua navega��o. Navega��o, conceitualmente, � o ato de guiar um rob� em um espa�o por um acaminho que possa ser percorrido levando o rob� de uma posi��o e orienta��o iniciais para uma posi��o e orienta��o finais.
---------------------------------------------------------------------
LIVRO
Neste projeto, a rob�tica � vista de uma perspectiva da ci�ncia de computa��o, no n�vel cognitivo do rob�. Cogni��o representa a decis�o e execu��o de tarefas para atingir um objetivo maior.

No caso de um rob� m�vel, o aspetco espec�fico de cogni��o diretamente ligado a uma mobilidade robusta � competencia de navega��o

---------------------------------------------------------------------

A proposta desse projeto � desenvolver uma aplica��o de rob�tica m�vel para o monitoramento de ambientes internos, onde normalmente, nesses ambientes, existem regi�es cr�ticas que possuem prioridades distintas para serem monitoradas. A prioridade (urg�ncia) da visita de cada uma dessas regi�es aumenta conforme o tempo passa e essa regi�o n�o � visitada. Ao ser visitada, a urg�ncia de uma determinada regi�o �, ent�o, diminu�da. Para a solu��o do problema descrito anteriormente, sup�es-se que o rob� tenha uma descri��o do ambiente em que atua (mapa). Existem diversas aplica��es pr�ticas para esse tipo de aplica��o. Dentre elas, pode-se citar o desenvolvimento de um rob� vigia que monitora um ambiente, dando �nfase para �reas cr�ticas. Outro exemplo consiste em um rob� que monitora a temperatura de ambientes onde esse fator � cr�tico.

Na literatura, um problema semelhante que vem sendo estudando h� muitos anos � o Problema do caixeiro-viajante da classe de problemas de roteamento de n�s, onde o caixeiro deve, dado um conjunto de cidades, partir de uma cidade base, visitar as outras somente uma vez e retornar � base \cite{Arenales2007}. 

\section{Motiva��o}
\label{motivacao}
A motiva��o deste projeto � encontrar uma estrat�gia e/ou um algoritmo eficiente para o monitoramento de ambientes internos, com a finalidade de ser utilizado em sistemas de seguran�a inteligentes, hospitais, ou qualquer outra situa��o em que o monitoramento de ambientes internos possa ser �til.

%A motiva��o deste projeto est� em expandir os conhecimentos nas �reas de navega��o e explora��o de ambientes internos, com utiliza��o de t�cnicas de intelig�ncia artificial, o interesse particular por rob�tica, o contato anterior com o rob� que ser� utilizado para os testes reais, a familiaridade com a ferramenta de desenvolvimento e com a ferramenta de simula��o Player/Stage \cite{Gerkey2003} e, tamb�m, a possibilidade de dar continuidade ao projeto iniciado na disciplina \textit{Projeto Supervisionado ou de Gradua��o I} e \textit{II}, colocando em pr�tica conceitos aprendidos nas disciplinas cursadas na gradua��o.

\section{Objetivo}
\label{objetivo}
Esse projeto tem como objetivo o desenvolvimento e compara��o entre estrat�gias e algoritmos para determinar uma seq��ncia de �reas a serem visitadas em ambientes internos com a finalidade de monitoramento desses ambientes utilizando um rob� m�vel. O problema a ser resolvido consiste na divis�o de um ambiente previamente conhecido em �reas de interesse. A cada uma dessas �reas � atribu�do um valor (peso) referente a sua import�ncia de monitoramento. A prioridade com que o rob� deve visitar determinadas �reas � calculada com base na import�ncia dessas �reas e no tempo decorrido desde a sua �ltima visita. �reas de maior import�ncia devem ser visitadas mais freq�entemente. 

As avalia��es consistem na compara��o dos algoritmos e estrat�gias. Ser�o utilizados dois crit�rios de compara��o, um crit�rio comparando a freq��ncia relativa de cada sala com sua prioridade relativa, onde o melhor resultado � aquele em que a freq��ncia relativa se aproximar mais da prioridade relativa. O segundo crit�rio � um gr�fico mostrando a progress�o da somat�ria dos graus de urg�ncia de todas as salas.

%\section{Organiza��o da Monografia}
%\label{organizacao_monografia}
%Organiza��o da monografia.

%==========================================================
% Rob�s M�veis.
% 
% Autor
% Heitor Luis Polidoro
% 
% Orientador
% Prof. Dr. Denis Fernando Wolf
%==========================================================

\chapter{Rob�s M�veis}
\label{robos}

Este cap�tulo apresenta as principais caracter�sticas dos Rob�s M�veis Aut�nomos. Um rob� m�vel aut�nomo � dotado de tr�s caracter�sticas importantes: Ativadores, respons�veis, entre outras coisas pela locomo��o; Sensores, que fazem o papel da percep��o do rob� e Controle, que aciona os ativadores de acordo com a leitura dos sensores, seguindo o prop�sito para qual o rob� est� sendo utilizado.

\section{Locomo��o}
\label{locomocao}
Rob�s m�veis precisam de mecanismos de locomo��o que os permitam moverem-se livremente pelo ambiente, mas existe uma grande variedade de modos de se locomover: andar, pular, correr, deslisar, patinar, nadar ou voar, utilizando rodas ou esteiras \cite{Siegwart2004}. A maioria dos mecanismos de locomo��o s�o inspirados na natureza, exceto os que usam rodas, esteiras ou propuls�o (aqu�tica ou a�rea).

Rob�s com rodas s�o os mais populares por in�meras raz�es. Eles s�o mecanicamente simples de construir. A raz�o da carga peso/mecanismo � favor�vel. Os outros mecanismos geralmente precisam de um hardware mais complexo para carregarem a mesma carga \cite{Jones1999}. S�o mais eficientes em gasto de pot�ncia por velocidade como mostra a Figura \ref{fig:potencia_por_velocidade}. Al�m disso, como os rob�s com rodas s�o projetados para tr�s ou mais rodas estarem em contato com o ch�o todo o tempo, o equil�brio n�o � usualmente um problema pesquisado \cite{Siegwart2004}.

\figura{../../imagens/grafico_potencia_por_velocidade.png}{.5}{Pot�ncia por velocidade de v�rios mecanismos de locomo��o: (a) Rastejar e deslizar. (b) Correr. (c) Pneu em ch�o suave. (d) Andar. (e) Em ferrovia. (1) Unidade de pot�ncia (Cavalos/toneladas). (2) Velocidade (milhas/hora)\cite{Siegwart2004}.}{fig:potencia_por_velocidade}

Ao inv�s de se preocuparem com o equil�brio, as pesquisas em rob�s com rodas tendem a focar em problemas como tra��o, estabilidade, manobrabilidade e controle \cite{Siegwart2004}. 

Existem quatro classes principais de rodas, como � mostrado na Figura \ref{fig:tipos_de_rodas}. Elas se diferenciam grandemente na sua cinem�tica, e portanto, a escolha da classe de roda tem um grande impacto na cinem�tica do rob� m�vel. A roda padr�o e a roda castor t�m um eixo prim�rio de rota��o e s�o portanto altamente direcionais. Para mover em uma dire��o diferente, a roda precisa ser direcionada primeiro ao longo do eixo vertical. A diferen�a entre essas duas rodas � que a roda padr�o consegue concluir seu direcionamento sem efeitos colaterais; porque o centro de rota��o passa pelo contato com o ch�o, ao passo que a roda castor rotaciona em torno de um eixo deslocado, causando uma for�a a ser transmitida para o chassi do rob� durante o direcionamento \cite{Siegwart2004}.

\figura{../../imagens/tipos_de_roda.png}{.6}{Quatro tipos b�sicos de roda. (a) Roda padr�o. (b) Roda castor. (c) Roda sueca. (d) Bola ou roda esf�rica\cite{Siegwart2004}.}{fig:tipos_de_rodas}

As fun��es da roda sueca s�o as de uma roda normal, mas prov�m baixa resist�ncia em outra dire��o, algumas vezes perpendicular � dire��o convencional, como a Sueca 90, e algumas vezes em um �ngulo intermedi�rio, como a Sueca 45. Os pequenos roletes ao redor da roda s�o passivos. A vantagem principal desse modelo � que enquanto a rota��o � provida somente atrav�s de um eixo principal, a roda pode cinematicamente mover-se com pouco atrito em diferentes poss�veis trajet�rias, n�o somente para frente e para tr�s \cite{Siegwart2004}.

A roda esf�rica � uma roda omnidirecional verdadeira, muitas vezes projetada de modo que ela possa ser alimentada ativamente para girar ao longo de qualquer dire��o. Um modo para implementar a roda esf�rica � imitar o \textit{mouse} do computador, provendo ativamente rolamentos alimentados que encostam na superf�cie superior da esfera e concedem for�a rotacional \cite{Siegwart2004}.

A principal desvantagem das rodas � que elas necessitam de uma rua ou uma superf�cie relativamente plana \cite{Braunl2008}, sendo que em terrenos desiguais, elas podem ter uma performance pobre. Genericamente, se a altura do objeto for aproximadamente o raio da roda a roda n�o consegue passar sobre esse objeto. Uma solu��o simples seria utilizar rodas grandes o suficiente, mais que o dobro da altura de quaisquer poss�veis obst�culos, mas em muitos casos isso � impratic�vel \cite{Jones1999}.

\section{Sensores}
\label{sensores}
Uma das tarefas mais importantes em um sistema aut�nomo de qualquer tipo � adquirir conhecimento sobre o ambiente. Isso � feito adquirindo medidas usando v�rios sensores e extraindo informa��es significativas dessas medidas \cite{Siegwart2004}. Existe um vasto n�mero de sensores sendo usados em rob�tica, aplicando diferentes t�cnicas de medidas e usando diferentes interfaces \cite{Braunl2008}. Como humanos, n�s conseguimos ver uma x�cara em cima da mesa e, sem pensar muito, conseguimos pegar essa x�cara. De fato, completar a simples tarefa de alcan�ar e erguer uma x�cara requer uma combina��o complexa de sensores, interpreta��o, cogni��o e coordena��o \cite{Jones1999}.

Alguns sensores s�o utilizados para medidas simples como a temperatura interna do rob� ou a velocidade de rota��o dos motores, mas outros mais sofisticados podem ser utilizados para adquirir informa��o sobre o ambiente ou at� para medir diretamente a posi��o global do rob� \cite{Siegwart2004}. 

Pode-se classificar os sensores em dois importantes eixos funcionais: proprioceptivo/exteroceptivo e passivo/ativo \cite{Siegwart2004}.
\textbf{Proprioceptivos} s�o os sensores que medem valores internos ao sistema, como velocidade do motor, voltagem bateria, �ngulo dos bra�os por exemplo.
\textbf{Exteroceptivos} s�o os sensores que adquirem informa��es sobre o ambiente do rob�, como medidas de dist�ncia, intensidade de luz e amplitude do som, por exemplo. Conseq�entemente, as medidas de sensores exteroceptivos s�o interpretadas pelo rob� para extrair as informa��es significantes do ambiente.
\textbf{Passivos} s�o sensores que medem par�metros do ambiente que entram no sensor. Por exemplo: sondas de temperatura, microfones e c�meras.
\textbf{Ativos} s�o sensores que emitem energia para o ambiente, ent�o medem a rea��o do ambiente � essa energia. Sensores deste tipo t�m uma performance superior, entretanto, existem alguns riscos, como a falta de energia, que pode afetar a caracter�stica que o sensor est� tentando medir. Sensores ativos tamb�m podem sofrer interfer�ncias de sinais que est�o fora do seu controle. Por exemplo, sinais emitidos por rob�s pr�ximos ou sensores similares no mesmo rob�, podem influenciar no resultado das medidas. Exemplos de sensores ativos incluem sensores ultra-s�nicos, sonares, e LASERs para medir dist�ncia \cite{Siegwart2004}. 

\subsection{Od�metro}
\label{odometro}
O od�metro � um sensor proprioceptivo que mede as rota��es das rodas e baseado nessa informa��o � poss�vel determinar a posi��o do rob� no mapa. O principal problema dos od�metros � com rela��o � sua precis�o, pois muitos fatores podem atrapalhar a leitura, como por exemplo, uma roda patinando ou um pneu murcho poder� fazer com que o od�metro interprete que o rob� esteja fazendo um movimento diferente do real. Para corrigir esse tipo de problema existem os algoritmos de localiza��o, que ser�o explicados em \ref{localizacao}.

\subsection{LASER}
\label{laser}
O LASER � um sensor Exteroceptivo Ativo, pois ele emite um sinal de luz e calcula a dist�ncia do objeto em rela��o ao sensor. Um emissor envia o sinal e calcula o tempo que esse sinal demora para retornar ao emissor, com isso � poss�vel calcular a dist�ncia naquela dire��o, e atrav�s de um espelho rotativo � poss�vel atingir at� $360^\circ$ de alcance. A Figura \ref{fig:laser} mostra um LASER $180^{\circ}$.

\begin{figure}[h]
	\centering
		\includegraphics[width=0.30\textwidth]{../../imagens/laser.png}
	\caption{Exemplo de Funcionamento do LASER}
	\label{fig:laser}
\end{figure}

\section{Rob� Pioneer}
\label{robo_pioneer}

O Pioneer 3 DX (Figura \ref{fig:pioneer}) � o rob� dispon�vel para testes experimentais no Laborat�rio de Rob�tica M�vel (ICMC - USP), ele � um rob� m�vel �gil e vers�til. Constru�do em um sistema cliente-servidor, ele oferece um processamento de vis�o \textit{onboard}, comunica��o \textit{ethernet}, LASER, GPS, sonar, e outras fun��es aut�nomas. Para program�-lo deve-se utilizar a biblioteca Player \cite{MobileRobots}.

O Pioneer 3 DX pode carregar at� 23kg, chega at� 1.6 m/s, possui rodas de 19cm, oito sonares na parte frontal e um computador interno com algumas fun��es j� implementadas, como a utilizada no projeto que � o desvio de obst�culos utilizando o algoritmo VFH (\textit{Vector Field Histogram} explicado na se��o \ref{VFH}).

\figura{../../imagens/pioneer.jpg}{.9}{Rob� Pioneer.}{fig:pioneer}


\section{Player}
\label{player}
O Player � um servidor de rede para controlar rob�s. Executando embarcado no rob�, o Player prov� uma interface simples e clara dos sensores e atuadores do rob� sobre uma rede IP. O programa cliente ``conversa'' com o Player utilizando sockets TCP, lendo dados dos sensores, escrevendo comandos nos atuadores e configurando dispositivos em tempo de execu��o.
O servidor Player foi desenvolvido para que seus clientes sejam independentes de linguagem e de plataforma. O programa cliente pode ser executado em qualquer m�quina que tenha conex�o de rede com o rob�, e pode ser escrito em qualquer linguagem que suporte sockets TCP. Atualmente existem clientes dispon�veis em C++, Tcl, Java e Phyton \cite{Player}. Futuramente o Player ser� indiferente sobre como o programa de controle do rob� � estruturado, ou seja, poder� escrever desde programas \textit{multi-threads} altamente concorrentes at� programas sequencias simples.

\subsection{Interface dos Drivers}
\label{player_interface}
Usualmente as interfaces entre o programa do usu�rio e o rob� se d� como mostra a Figura \ref{fig:interface_usual}, onde o programa do usu�rio tem que fazer a aquisi��o de dados diretamente dos sensores e tratar esses dados de modo que o planejador compreenda. E gerar os comandos para o motor ou outros atuadores. Sendo que para cada modelo de motor, os comandos s�o diferentes, assim como cada c�mera, LASER e outros sensores t�m leituras diferentes de acordo com o modelo.

\begin{figure}[hb]
	\centering
		\includegraphics[width=1.00\textwidth]{../../imagens/player1.png}
	\caption{Interface usual.}
	\label{fig:interface_usual}
\end{figure}

O Player tem como fun��o fazer essa interface entre os sensores e atuadores e o programa do usu�rio (Figura \ref{fig:player_interface}), adquirindo os dados de qualquer modelo de um mesmo sensor (que seja aceito pelo Player) e fornecendo ao programa do usu�rio os dados j� tratados, ou seja, para qualquer modelo de um mesmo sensor (que seja aceito pelo Player) o programa do usu�rio l� da mesma forma, assim como para os atuadores como mostra a Figura \ref{fig:modelo_cliente_servidor}. Por exemplo, para fazer com que o rob� ande para frente na velocidade de 1 m/s, basta o programa do usu�rio informar a dire��o e velocidade, o comando vai ser o mesmo para diversos modelos de rob�s, que possuem tamanhos e n�meros de rodas diferentes, o que realiza os c�lculos para determinar qual a rota��o da roda para atingir tal velocidade � a interface do Player, assim como para ler do sensor LASER, basta informar de qual �ngulo quer ler.

\begin{figure}[hb]
	\centering
		\includegraphics[width=1.00\textwidth]{../../imagens/player2.png}
	\caption{Interface do Player}
	\label{fig:player_interface}
\end{figure}

\begin{figure}[hb]
	\centering
		\includegraphics[width=1.00\textwidth]{../../imagens/modelo_cliente_servidor.png}
	\caption{Interface e Modelo Cliente-Servidor}
	\label{fig:modelo_cliente_servidor}
\end{figure}

A Figura \ref{fig:modelo_cliente_servidor} tamb�m ilustra o funcionamento do modo Cliente-Servidor do Player, como a comunica��o � feita por meio de \textit{sockets} o programa cliente pode ser escrito em qualquer ling�agem que permita comunica��o atrav�s de \textit{sockets}. Esse modo Cliente-Servidor tamb�m permite a f�cil utiliza��o de diversos simuladores, pois como a comunica��o � a mesma, o programa cliente n�o sabe se est� interagindo com o rob� real ou com algum simulador (Figura \ref{fig:abstracao_de_hardware}).

\begin{figure}[hb]
	\centering
		\includegraphics[width=1.00\textwidth]{../../imagens/abstracao_de_hardware.png}
	\caption{Abstra��o de Hardware}
	\label{fig:abstracao_de_hardware}
\end{figure}

Esse modelo Cliente-Servidor tamb�m tr�s outras vantagens como:

\begin{itemize}
	\item Clientes podem se conectar a m�ltiplos servidores;
	\item Servidores aceitam conex�o de m�ltiplos clientes;
	\item Diferentes programas/processos/\textit{threads} podem processar dados de diferentes sensores do mesmo servidor.
\end{itemize}

A Figura \ref{fig:exemplo_cliente_servidor} ilustra esse funcionamento, onde a letra 'P' indica que naquele dispositivo existe um programa Player sendo executado e a letra 'C' indica que naquele dispositivo est� sendo executado um programa cliente.

\begin{figure}[hb]
	\centering
		\includegraphics[width=1.00\textwidth]{../../imagens/exemplo_cliente_servidor.png}
	\caption{Exemplo de Clientes e Servidores}
	\label{fig:exemplo_cliente_servidor}
\end{figure}

\section{Stage}
\label{stage}
O Stage � usado normalmente como um \textit{plugin }para o Player, provendo uma s�rie de dispositivos virtuais para os clientes Player. Os usu�rios escrevem as rotinas e algoritmos normalmente, como clientes para um servidor Player. N�o � poss�vel para clientes distinguir a diferen�a entre os dispositivos reais do rob� e os equivalentes simulados pelo Player/Stage. Com isso clientes Player desenvolvidos usando o Stage precisar�o de pouca ou nenhuma modifica��o para trabalhar com o rob� real e vice-versa. Em muitos casos, basta somente mudar no cliente o endere�o IP de onde est� o servidor. O Stage tamb�m pode simular uma popula��o de rob�s m�veis, sensores e objetos num ambiente bidimensional (Figura \ref{fig:stage}) \cite{Player}. Nesta disserta��o ser� utilizado somente um rob� e o sensor LASER.

\figura{../../imagens/stage.jpg}{.5}{Simula��o com 5 rob�s, 2 objetos, LASER, sonar e \textit{blobfinder}.}{fig:stage}

%========================================================== 
% Navega��o de Rob�s M�veis.
% 
% Autor Heitor Luis Polidoro
% 
% Orientador Prof.  Dr.  Denis Fernando Wolf
%==========================================================

\chapter{Navega��o de Rob�s M�veis} 
\label{navegacao} 
Existem diversos problemas em aberto na rob�tica m�vel.  Uma dessa quest�es ainda n�o resolvida consiste em um rob� com a habilidade de navegar em seu ambiente \cite{Jones1999}.  A navega��o � a ci�ncia, arte, pr�tica ou tecnologia, de planejar e percorrer uma trajet�ria de um ponto de origem at� um ponto de destino. 

Em rob�tica, navega��o � a ci�ncia de direcionar o percurso de um rob� m�vel enquanto percorre o meio ambiente (terra, �gua, ou ar).  Inerente em qualquer esquema de navega��o � o desejo de alcan�ar um destino sem se perder ou colidir com algum obst�culo \cite{Mckerrow1991}.  Em geral, navega��o � um processo incremental que, segundo \cite{Murphy2000}, pode ser resolvido respondendo � quatro perguntas:

\begin{itemize} 
	\item \textbf{Para onde estou indo?} Geralmente determinado por um humano ou uma miss�o; 
	\item \textbf{Qual o melhor caminho?} Esse � o problema de planejamento de trajet�ria, e � a �rea da navega��o que recebe mais aten��o; 
	\item \textbf{Por onde passei?} Enquanto o rob� explora o ambiente, pode ser parte da miss�o mapear esse ambiente; 
	\item \textbf{Onde estou?} Para seguir uma trajet�ria ou construir um mapa o rob� precisa saber onde ele est�.  
\end{itemize}

Que podem ser sumarizadas em quatro passos, segundo \cite{Goldberg1995}:

\begin{itemize} 
	\item Percep��o e modelagem do ambiente; 
	\item Localiza��o; 
	\item Planejamento e decis�o do movimento; 
	\item Execu��o do movimento; 
\end{itemize}

A rela��o entre esses passos pode ser vista na Figura \ref{fig:fluxo_navegacao} \cite{Mckerrow1991}.

\figura{imagens/fluxo_navegacao.jpg}{.4}{Hierarquia do controle de um rob� m�vel, mostrando o fluxo de informa��o.}{fig:fluxo_navegacao}

Navega��o � a inst�ncia do paradigma geral da rob�tica ``perceber - decidir - agir''.  A implementa��o da tarefa de navega��o pode ser mais ou menos complexa, depende do contexto em que a tarefe vai ser executada \cite{Goldberg1995}.

\begin{itemize} 
	\item \textbf{O ambiente:} pode ser inicialmente conhecido, parcialmente conhecido, ou completamente desconhecido, pode ser est�tico ou com objetos m�veis.; 
	\item \textbf{A meta:} pode ser especificada por \textit{landmarks} ou coordenadas; 
	\item \textbf{A navega��o em si:} pode ter restri��es como: tempo, melhor caminho.; 
	\item \textbf{As habilidades do rob�:} poder de computa��o, sensores e suas incertezas, tamanho do rob� e sua cinem�tica,.  
\end{itemize}

A solu��o para o problema de navega��o vai depender de todas essas restri��es \cite{Goldberg1995}.

A navega��o pode ser dividida em duas grandes �reas: planejamento de trajet�ria e desvio de obst�culos.  O planejamento de trajet�ria consistem em, dada uma representa��o do ambiente (total ou parcial) o rob� deve planejar uma trajet�ria que o leve do seu ponto de origem at� seu destino, atendendo a requisitos como menor caminho ou  menos curvas por exemplo.  O desvio de obst�culos � utilizado principalmente em ambientes din�micos, onde possam haver obst�culos m�veis, e sua fun��o � fazer com que o rob� chegue em seu destino de forma segura, ou seja, n�o colida com obst�culos que podem ser m�veis ou n�o, utilizando sensores geralmente de dist�ncia, como LASERs e sonares, ou at� mesmo utilizando c�meras.

\section{Localiza��o}
\label{localizacao}
FALAR SOBRE

\section{Representa��o do Mapa}
\label{mapa}
Representar o ambiente onde o rob� se encontra � importante pois, decis�es baseadas na representa��o do ambiente podem ter impacto nas escolhas dispon�veis para a representa��o da posi��o do rob�. Muitas vezes a fidelidade da representa��o da posi��o � limitada pela fidelidade do mapa \cite{Siegwart2004}.

Segundo \cite{Siegwart2004}, tr�s rela��es fundamentais tem de ser entendidas quando se escolhe uma representa��o particular de mapa:
\begin{itemize}
	\item A precis�o do mapa precisa ser compat�vel com a precis�o da necessidade do rob� para atingir seus objetivos;
	\item A precis�o do mapa e o tipo de dados dos recursos representados precisam ser compat�veis com a precis�o e os tipo de dados retornados pelos sensores do rob�;
	\item A complexidade da representa��o do mapa tem impacto direto com a complexidade computacional.	
\end{itemize}

Existem v�rias maneira de representar um mapa: Decomposi��o por c�lula, decomposi��o fixa, decomposi��o adaptada (com varia��o de c�lula), \textit{occupancy grid} e representa��o topol�gica, et al. \cite{Siegwart2004}.

EXPLICAR A DIFEREN�A ENTRE MAPAS M�TRICOS E TOPOL�GICOS, COLOCAR FIGURAS

\section{Planejamento de Trajet�ria} 
\label{planejamento} 
O primeiro passo para planejar uma trajet�ria � transformar um poss�vel modelo cont�nuo do ambiente em um mapa discreto compat�vel com o algoritmo de planejamento de trajet�ria escolhido.  � poss�vel identificar tr�s estrat�gias gerais de composi��o \cite{Siegwart2004}: 

\begin{itemize} 
	\item \textit{Roadmap}: identificar um conjunto de rotas nos espa�os livres; 
	\item Decomposi��o em c�lulas: discriminar entre c�lulas livres e ocupadas; 
	\item Campos potenciais: impor uma fun��o matem�tica sobre o espa�o.  
\end{itemize}

\subsection{\textit{Roadmap}} 
\label{roadmap} 
A t�cnica de \textit{roadmap} para planejamento de trajet�rias consiste em capturar a conectividade do espa�o livre do ambiente em uma rede de curvas.  Essa rede � vista como um conjunto padr�o de caminhos.  O planejamento da trajet�ria ent�o se reduz � conectar os pontos inicial e final do rob� no \textit{roadmap} e buscar neste um caminho entre esses dois pontos. Se existir um caminho ele ser� dado pela jun��o de tr�s subcaminhos: um subcaminho entre o ponto inicial at� algum ponto do \textit{roadmap}, um subcaminho do \textit{roadmap} e um subcaminho do \textit{roadmap} at� o ponto final \cite{Ottoni2000}

V�rios m�todos propostos foram baseados nessa id�ia, dentre eles: grafos de visibilidade, diagrama de Voronoi, rede de caminho livre, e silhueta.  Nesta disserta��o ser� utilizado o diagrama de Voronoi. 

\paragraph{Diagrama de Voronoi} 
\label{voronoi} 
Diagrama de Voronoi � um m�todo completo de mapa de rotas que tende � maximizar a dist�ncia enter o rob� e os obst�culos no mapa.  Para cada ponto livre no mapa � calculada a dist�ncia para o obst�culo mais pr�ximo. O diagrama de Voronoi consiste nos pontos que s�o eq�idistantes de um ou mais obst�culos.  Um exemplo de diagrama de Voronoi em um mapa pode ser visto na Figura \ref{fig:exemplo_voronoi} \cite{Siegwart2004}.

\figura{imagens/exemplo_voronoi.jpg}{.4}{Exemplo de um diagrama de Voronoi.\cite{Siegwart2004}}{fig:exemplo_voronoi}

O diagrama de Voronoi tem uma fraqueza importante no caso sensores de localiza��o com alcance limitado.  Como o algoritmo maximiza a dist�ncia entre o rob� e um objeto no ambiente, qualquer sensor de curto alcance no rob� poder� falhar para perceber ao seu redor. Se o sensor de curto alcance est� sendo usado para localiza��o, ent�o o caminho designado pelo diagrama de Voronoi ser� pobre num ponto de vista para localiza��o \cite{Siegwart2004}.

Por outro lado, como por defini��o o caminho � criado baseado em um ponto eq�idistante dos obst�culos, isso garante um rota segura do rob� pelo mapa.

\paragraph{Decomposi��o em C�lulas} 
\label{decomposicao_celulas} 
Este m�todo consiste em dividir o espa�o livre do mapa em c�lulas, de forma que um caminho entre quaisquer duas c�lulas possa ser facilmente obtido. Um grafo, chamado \textit{grafo de conectividade}, representa a rela��o de adjac�ncia entre as c�lulas.  Onde os v�rtices representam as c�lulas extra�das do espa�o livre.  Somente existe uma aresta entre dois v�rtices se e somente se as c�lulas correspondentes foram adjacentes.  O resultado de um caminho � uma seq��ncia de c�lulas denominada \textit{canal}, de onde pode ser computado um caminho cont�nuo \cite{Ottoni2000}.  A Figura \ref{fig:decomposicao_celulas} demonstra um exemplo de de mapa utilizando decomposi��o em c�lulas.

\figura{imagens/decomposicao_celulas.jpg}{.5}{Exemplo de um mapa com decomposi��o em c�lulas.}{fig:decomposicao_celulas}

\paragraph{Campos Potenciais} 
\label{campos_potenciais}
Neste m�todo o espa�o livre � discretizado em uma fina grade, a cada posi��o � associada uma fun��o, com a qual pode-se fazer uma analogia a um campo potencial.  Como o tamanho da grade � grande, pois ela � fina, s�o utilizados m�todos heur�sticos para encontrar um caminho. A analogia que o m�todo sugere � que o rob� seja uma part�cula movendo-se sob a influ�ncia de um campo potencial gerado pelos obst�culos e pelo seu ponto de destino. O ponto de destino gera um campo potencial que atrai o rob�, enquanto os obst�culos geram um campo que repele o rob�. O caminho final � dado pela for�a resultante desses campos potenciais. 

Um exemplo de campos potenciais � mostrado na Figura \ref{fig:campos_potenciais}.  O campo potencial atrativo (b) � um parabol�ide com ponto de m�nimo localizado na posi��o do objetivo.  O campo potencial repulsivo (c) � diferente de zero somente a partir de uma determinada dist�ncia dos obst�culos. O caminho (e) � constru�do pela dire��o oposta a do gradiente do potencial resultante (d). Em (f) temos uma matriz de orienta��es do vetor gradiente, que s�o as orienta��es das for�as induzidas pelo campo potencial \cite{Ottoni2000}.

\figura{imagens/campos_potenciais.jpg}{.5}{Exemplo de um mapa com campos potenciais.\cite{Ottoni2000}}{fig:campos_potenciais}

\section{Desvio de Obst�culos} 
\label{desvio} 
Recentemente, muitas pesquisas voltaram sua aten��o para o problema de desvio de obst�culos.  Entretanto, existem alguns m�todos cl�ssicos de desvio de obst�culos que dever ser citados \cite{Borenstein1991}: Detec��o de borda, \textit{certainty grid}, campos potenciais, campo de for�a virtual e vector field histogram. 

Detec��o de borda, � um m�todo bem popular que extrai as bordas verticais do objeto, e guia o rob� ao redor dessas bordas. 

\textit{Certainty grid} � um m�todo de representa��o probabil�stico de obst�culos que modela o mundo em uma grade, onde a �rea de trabalho do rob� � modelada em um arranjo de quadrados em 2D, chamadas de c�lulas.  Cada c�lula tem um valor de certeza que indica o grau de confian�a de que algum objeto est� na �rea dessa c�lula. 

O m�todo de campos potenciais funciona tanto para planejamento de trajet�ria quanto para desvio de obst�culos, para isso basta calcular a for�a potencial resultante em tempo de execu��o, com isso o rob� poder� desviar de obst�culos m�veis

O campo de for�a virtual (do ingl�s \textit{Virtual Force Field} - VFF) � um m�todo para ve�culos que necessitam de uma resposta mais r�pida para fazer curvas. � um m�todo baseado na \textit{certainty grid}, onde uma grade de histograma cartesiano 2D � usado para representar a probabilidade de cada c�lula conter um obst�culo, depois a id�ia de campos potenciais � aplicada ao histograma.

E por fim o m�todo de \textit{vector field histogram} - VFH, que cria um mapa de \textit{certainty grid} local, e ao inv�s de utilizar um histograma cartesiano 2D, utiliza um histograma polar ($\alpha-P$), onde $\alpha$ � o �ngulo do sensor e $P$ � a probabilidade de haver um obst�culo nessa dire��o. O VFH � explicado melhor a seguir.

O m�todo utilizado foi o VFH, por j� vir implementado no Player e em testes se mostrou suficiente para o projeto.

\subsection{\textit{Vector Field Histogram}}
\label{VFH}
Analisando melhor o m�todo VFF percebemos um problema: redu��o dr�stica excessiva dos dados quando for�as de repuls�o individuais do histograma da grade de c�lulas s�o somadas para calcular o vetor resultante. Centenas de pontos de dados s�o reduzidos em um passo para dire��o e magnitude. Conseq�entemente, informa��es detalhadas sobre a distribui��o local do obst�culo � perdida. Para remediar essa situa��o, foi desenvolvido um novo m�todo chamado \textit{Vector Field Histogram} - VFH. O M�todo VFH utiliza dois est�gios de redu��o de dados \cite{Borenstein1991}.

Existem tr�s n�veis de representa��o de dados \cite{Borenstein1991}:

\begin{itemize}
	\item O n�vel mais alto cont�m uma descri��o detalhada do ambiente do rob�;
	\item No n�vel intermedi�rio � armazenado um histograma polar unidimensional sobre a localiza��o moment�nea do rob�;
	\item O n�vel mais baixo � a sa�da do algoritmo VFH, valores de refer�ncia para o controle de movimento do rob�
\end{itemize}

No primeiro est�gio da redu��o de dados � mapeada a grade de histograma do ambiente (Figura \ref{fig:histogram_grid}) em um histograma polar (Figura \ref{fig:polar_histogram}). E no segundo est�gio computa a dire��o $\theta$ para onde o rob� deve virar.

\figura{imagens/histogram_grid.png}{.3}{(a) Exemplo de um mapa. (b) Grade de histograma correspondente.\cite{Borenstein1991}}{fig:histogram_grid}

\figura{imagens/polar_histogram.png}{.3}{(a) Densidade polar de obst�culo em um histograma polar relativo � posi��o do rob� em $O$ (Figura \ref{fig:histogram_grid}(b)). (b) O mesmo histograma polar de (a) mostrado em forma polar e por cima da grade de histograma da Figura \ref{fig:histogram_grid}(b).\cite{Borenstein1991}}{fig:polar_histogram}

EXPLICAR COM MAIS DETALHES, VER DISSERTA��O DO VANDERSON

%==========================================================
% Grafos e aplica��es.
% 
% Autor
% Heitor Luis Polidoro
% 
% Orientador
% Prof. Dr. Denis Fernando Wolf
%==========================================================
\chapter{Grafos e Aplica��es}
\label{grafos_aplicacoes}
Grafos e mapas topol�gicos, s�o muito utilizados na literatura para navega��o de rob�s m�veis, como pode ser constatado em \cite{Adorno2005,Scatena2004,Thrun1996,Thrun1998,Thrun1998a,Neto2003}. Como o problema proposto (se��o \ref{resultados}) diz respeito � busca de trajet�ria em um mapa topol�gico (grafo), uma das solu��es se torna uma varia��o do problema do caixeiro viajante que trata, basicamente, de encontrar o menor ciclo hamiltoniano em um grafo.

As se��es seguintes cont�m uma explica��o sobre grafos, mapas topol�gicos, ciclo hamiltoniano e o problema do caixeiro viajante.

\section{Grafos}
\label{grafos}
Muitas aplica��es em computa��o necessitam considerar um conjunto de conex�es entre objetos. Os relacionamentos dessas conex�es podem ser utilizados para responder quest�es como: Existe caminho de um objeto a outro? Quantos objetos podem ser alcan�ados a partir de um determinado objeto? Qual a menor dist�ncia entre dois objetos? Existe um tipo abstrato de dados chamado grafo que � usado para modelar essas situa��es \cite{Ziviani2004}.

Um grafo � constitu�do de um conjunto de v�rtices e um conjunto de arestas que conectam pares de v�rtices. Um v�rtice � um objeto que pode conter nomes e outros atributos. Os grafos podem ser direcionados ou n�o direcionados. Um grafo direcionado $G$ � um par $(V, A)$, em que $V$ � um conjunto finito de v�rtices e $A$ � um conjunto de arestas com uma rela��o bin�ria. A Figura \ref{fig:grafo_dir} mostra um grafo direcionado com o conjunto de v�rtices $V = {0, 1, 2, 3, 4, 5}$ e de arestas $A = {(0, 1), (0, 3), (1, 2), (1, 3), (2, 2), (2, 3), (3, 0), (5, 4)}$. Em um grafo n�o direcionado as arestas $(u, v)$ e $(v, u)$ s�o consideradas as mesmas. A Figura \ref{fig:grafo_ndir} mostra um grafo n�o direcionado com o conjuntos de v�rtices $V = {0, 1, 2, 3, 4, 5}$ e de arestas $A = {(0, 1), (0, 2), (1, 2), (4, 5)}$. Em grafos direcionados podem existir arestas de um v�rtice para ele mesmo, chamadas de \textit{self-loops}, como a aresta $(2, 2)$ no grafo direcionado da Figura \ref{fig:grafo_dir} \cite{Ziviani2004}.

\begin{figure}[ht]
	\centering
	\subfigure[Grafo direcionado]{
		\includegraphics[width=0.2\textwidth]{../../imagens/grafo_direcionado.jpg}
		\label{fig:grafo_dir}}
	\subfigure[Grafo n�o direcionado]{
		\includegraphics[width=0.2\textwidth]{../../imagens/grafo_nao_direcionado.jpg}
		\label{fig:grafo_ndir}}
	\caption{Exemplo de grafos\cite{Ziviani2004}}
	\label{fig:exemplo_de_grafos}
\end{figure}

Em um grafo direcionado, a aresta $(u, v)$ sai do v�rtice $u$ e entra no v�rtice $v$. Se $(u, v)$ � uma aresta do grafo $G = (V, A)$, ent�o o v�rtice $v$ � adjacente ao v�rtice $u$. Quando o grafo n�o � direcionado, a rela��o de adjac�ncia � sim�trica \cite{Ziviani2004}.

Em um grafo, um caminho $(v_0, v_1, \cdots, v_k)$ forma um ciclo se $v_0 = v_k$ e o caminho cont�m pelo menos uma aresta. O ciclo � simples se os v�rtices $v_1, v_2, \cdots, v_k$ s�o distintos. O \textit{self-loop} � um ciclo de tamanho 1. Na Figura \ref{fig:grafo_dir}, o caminho $(0, 1, 2, 3, 0)$ forma um ciclo. Dois caminhos $(v_0, v_1, \cdots, v_k)$ e $(v'_0, v'_1, \cdots, v'_k)$ formam o mesmo ciclo se existir um inteiro $j$ tal que $v'_i = v_{(j+i) mod k}$ para $i = 0, 1, \cdots, k - 1$. Na Figura \ref{fig:grafo_dir}, o caminho $(0, 1, 3, 0)$ forma o mesmo ciclo que os caminhos $(1, 3, 0, 1)$ e $(3, 0, 1, 3)$. Um grafo sem ciclos � um grafo ac�clico \cite{Ziviani2004}.

Um grafo $G$ � definido como \textit{Hamiltoniano} se possui um ciclo contendo todos os v�rtices de $G$. Esse nome foi dado, porque, em 1856, Willian Rowan Hamilton inventou um jogo matem�tico que consistia em um dodecaedro no qual cada um dos vinte v�rtices recebeu o nome de uma cidade. O objetivo do jogo era viajar pelas arestas do dodecaedro, visitando cada cidade exatamente uma vez e retornando para o ponto inicial \cite{Beineke1978}. A Figura \ref{fig:dodecaedro_de_hamilton} mostra uma solu��o para o jogo.

\figura{../../imagens/dodecaedro_de_hamilton.jpg}{.5}{Dodecaedro de Hamilton.\cite{Beineke1978}}{fig:dodecaedro_de_hamilton}

Defini��es segundo Gondran, Minoux e Vajda \cite{Gondran1984}:

\begin{itemize}
	\item Um caminho passando somente uma vez em cada v�rtice de $G$ � chamado Caminho Hamiltoniano e tem comprimento de $N - 1$
	\item Um ciclo Hamiltoniano � um ciclo que passa somente uma vez em cada v�rtice de $G$ e tem comprimento $N$.
\end{itemize}

Um grafo n�o direcionado � conectado se cada par de v�rtices est� conectado por um caminho. Um grafo direcionado � fortemente conectado se cada dois v�rtices quaisquer s�o alcan��veis a partir um do outro. Um grafo ponderado possui pesos associados �s suas arestas. Esses pesos podem representar, por exemplo, custos ou dist�ncias. Um grafo completo � um grafo no qual todos os pares de v�rtices s�o adjacentes \cite{Ziviani2004}.

\section{Busca do Menor Caminho}
\label{busca_do_menor_caminho}
%
\subsection{Algoritmo de Dijkstra}
\label{algoritmo_de_dijkstra}
Existem diversos algoritmos para buscas em grafo na literatura \cite{Cormen2001}. O algoritmo de Dijkstra utiliza a t�cnica do relaxamento, que nada mais � que verificar se � poss�vel melhorar o caminho obtido at� o momento passando por um v�rtice diferente. O algoritmo de Dijkstra apresenta uma solu��o $O([m + n] � log(n))$ para a determina��o do menor caminho \cite{Dudek2000}, e � composto por tr�s passos:

\begin{itemize}
	\item Passo 1: Iniciar os valores:

\begin{algorithmic}
	\FORALL {$v \in V[G]$} 
	\STATE$	d[v] \gets \infty$
	\STATE$ \pi[v] \gets  nulo$
	\ENDFOR
	\STATE	$d[s] \gets 0$
\end{algorithmic}

	$V[G]$ � o conjunto de v�rtices $v$ que forma o grafo $G$.
	
	$d[v]$ � o vetor de dist�ncias do v�rtice $s$ at� cada v�rtice $v$.
	
	$\pi[v]$ identifica o v�rtice de onde se origina uma conex�o at� $v$ de maneira a formar um caminho m�nimo;

	\item Passo 2: Tem-se que usar dois conjuntos: $S$, que representa todos os v�rtices $v$ onde $d[v]$ j� cont�m o custo do menor caminho e $Q$ que cont�m os v�rtices restantes;

	\item Passo 3: Realiza-se uma s�rie de relaxamentos das arestas:
\begin{algorithmic}
	\WHILE {$Q \neq \emptyset $}
		\STATE$u \gets$ \texttt{extraia-m�n($Q$)}
		\STATE $S \gets S \cup {u}$
		\FORALL{$v$ adjacente a $u$}
			\STATE se $d[v] > d[u] + w(u, v)$ ent�o 
			\STATE $d[v] \gets d[u] + w(u, v)$
			\STATE $\pi[v] \gets u$
		\ENDFOR
	\ENDWHILE
\end{algorithmic}
		$w(u, v)$ � o peso da aresta que vai de $u$ a $v$.
		$u$ e $v$ s�o v�rtices quaisquer e $s$ � o v�rtice inicial.
		\texttt{extraia-m�n($Q$)}, retorna o menor elemento.
		
\end{itemize}

\section{Caixeiro-Viajante}
\label{caixeiro}

O problema do caixeiro-viajante envolve um conjunto de cidades e � da classe de problemas de roteamento de n�s, onde um caixeiro sai de uma cidade base, visita todas as cidades somente uma vez, e retorna � cidade base, otimizando um ou mais objetivos. Problemas de caixeiro-viajante s�o definidos em grafos orientados ou n�o orientados \cite{Arenales2007}.

A defini��o do problema do caixeiro-viajante �: Considera-se um grafo n�o orientado $G = (N, E)$, em que o conjunto $N$ consistem em $n$ cidades e $E$ representa o conjunto de arestas entre essas cidades. Supondo que $G$ � um grafo completo, isto �, para qualquer par de cidades $i, j \in \mathbb{N}, i \neq j$, existe uma aresta $(i, j)$. A dist�ncia entre as cidades $i$, e $j$ � $c_{ij}$, e quando $c_{ij} = c_{ji}$, o problema � dito sim�trico. Um caixeiro deve visitar $n$ cidades, passando por cada cidade somente uma vez, e retornar � cidade de partida. Esse percurso � denominado ciclo Hamiltoniano do grafo $G$, e o problema consiste em determinar o ciclo Hamiltoniano, ou rota, de dist�ncia m�nima. Devido � sua aplica��o em diversas �reas, este � um dos problemas combinat�rios mais pesquisados \cite{Arenales2007}.

Define-se as vari�veis

\begin{center}
$
x_{ij}=\left\{ \begin{array}{l}
			1  \mbox{ se o caixeiro vai diretamente da cidade $i$ � cidade $j$}, i \neq j \\
			0  \mbox{ se o caixeiro n�o vai da cidade $i$ � cidade $j$}, i \neq j
		\end{array}\right.
$
\end{center}

E considera-se o seguinte modelo:

\begin{equation}
	\mbox{min}\sum_{i=1}^n \sum_{j > i}c_{ij}x_{ij} 
	\label{eq:cv-func_obj}
\end{equation}

\begin{equation}
	\sum_{j < i} x_{ji} + \sum_{j > i} x_{ij} = 2, i = 1, \cdots, n
	\label{eq:cv-restr1}
\end{equation}

\begin{equation}
	x \in B^{n(n-1)/2}
	\label{eq:cv-restr2}
\end{equation}

A fun��o objetivo (\ref{eq:cv-func_obj}) expressa a minimiza��o da dist�ncia total da rota, e a restri��o (\ref{eq:cv-restr1}) imp�e que cada cidade tenha somente uma cidade sucessora imediata e uma cidade predecessora imediata, ou seja, � visitada uma �nica vez. Uma solu��o para o modelo anterior pode gerar sub-rotas desconexas (Figura \ref{fig:sub-rotas}) \cite{Arenales2007}.

\figura{../../imagens/sub-rotas.jpg}{.4}{Exemplo de poss�veis sub-rotas.\cite{Arenales2007}}{fig:sub-rotas}

Seja $S$ uma sub-rota, por exemplo, $S = {1. 2. 3. 4}$ na Figura \ref{fig:sub-rotas}. A elimina��o de sub-rotas pode ser obtida atrav�s da restri��o:
\begin{equation}
	\sum_{i \in S} \sum_{i \ in S \\ j > i} \leq |S| - 1, S \subset \mathbb{N}, 3 \leq |S| \leq \left\lfloor \frac{n}{2} \right\rfloor
	\label{eq:cv-el_sub-rotas1}
\end{equation}
Que garante que, para cada conjunto $S$, existem no m�nimo duas arestas que entram e/ou saem de $S$, ou seja, existem no m�nimo duas arestas entre cidades de $S$ e cidades fora de $S$. A cardinalidade de $S$ � no m�nimo 3 (pois ciclo em um grafo n�o orientado tem pelo menos 3 v�rtices) e no m�ximo $\left\lfloor\frac{n}{2}\right\rfloor$, pois ao se eliminar ciclos com $k$ v�rtices, elimina-se ciclos com $n - k$ v�rtices. A sub-rota $S = {1, 2, 3, 4}$ � eliminada.
\begin{center}
	$x_{15} + x_{16} + x_{17} + x_{25} + x_{26} + x_{27} + x_{35} + x_{36} + x_{37} + x_{45} + x_{46} + x_{47} \geq 2$
\end{center}


Como o n�mero de subconjuntos distintos de um conjunto de cardinalidade $k$ � $2^k$, a restri��o \ref{eq:cv-el_sub-rotas1} %e \ref{eq:cv-el_sub-rotas2}
t�m cardinalidade da ordem de $2^k$, $k \geq 6$, ou seja, o crescimento � exponencial em fun��o do n�mero de cidades. Para $k \leq 5$, a restri��o \ref{eq:cv-restr1} elimina sub-rotas \cite{Arenales2007}.

\section{Considera��es}
\label{consideracoes_grafo}

Como o mapa que � utilizado no projeto � um mapa baseado em grafo (mapa topol�gico), � poss�vel utilizar algumas teorias de grafos presentes na literatura como base para se achar uma solu��o para o problema. Assim como, foi utilizado o algoritmo de Dijkstra nas buscas de menor caminho para o rob� fazer a menor trajet�ria de um v�rtice a outro, e a fundamenta��o matem�tica da teoria do caixeiro-viajante que foi utilizada como base para se achar uma solu��o \textit{offline}.
%==========================================================
% Resultados Obtidos.
% 
% Autor
% Heitor Luis Polidoro
% 
% Orientador
% Prof. Dr. Denis Fernando Wolf
%==========================================================

\chapter{Desenvolvimento e Resultados}
\label{resultados_obitdos}
Esta disserta��o tem como objetivo o desenvolvimento de uma estrat�gia eficiente para determinar uma seq��ncia de �reas a serem visitadas em ambientes internos com a finalidade de monitoramento destes ambientes utilizando um rob� m�vel. O problema a ser resolvido consiste na divis�o de um ambiente previamente conhecido em �reas de interesse. A cada uma dessas �reas � atribu�do um valor (peso) referente � sua import�ncia de monitoramento. A prioridade com que o rob� deve visitar determinadas �reas � calculada com base na import�ncia dessas �reas e no tempo decorrido desde a sua �ltima visita. �reas de maior import�ncia devem ser visitadas mais freq�entemente. 

Para determinar a trajet�ria do rob� foram criadas duas solu��es, uma \textit{Offline} (se��o \ref{offline}) e outra em Tempo Real se��o (\ref{tempo_real}), essas solu��es ser�o confrontadas em dois crit�rios de avalia��o (se��o \ref{criterios}), um baseado no Grau de Urg�ncia Total e o outro baseado na freq��ncia relativa de cada sala.

\section{Crit�rios de Avalia��o}
\label{criterios}
Os algoritmos e estrat�gias ser�o ent�o testados e avaliados. As avalia��es consistem na compara��o dos algoritmos e estrat�gias. Ser�o utilizados dois crit�rios de compara��o, um crit�rio comparando a freq��ncia relativa de cada sala com sua prioridade relativa, onde o melhor resultado � aquele em que a freq��ncia relativa se aproximar mais da prioridade relativa, pois se uma sala (a) possui uma prioridade com valor duas vezes maior que uma sala (b), a sala (a) deve ser visitada com o dobro de freq��ncia do que a sala (b). Esse c�lculo � feito somando as diferen�as quadr�ticas entre a prioridade relativa e a freq��ncia relativa de cada sala ($\sum_{i=0}^n (P_i/P_t - F_i/F_t)^2$). Isto indica que o algoritmo ou estrat�gia se manteve fiel � defini��o do problema: \textit{ �reas de maior import�ncia devem ser visitadas mais frequentemente}.

O segundo crit�rio � um gr�fico mostrando a progress�o da somat�ria dos graus de urg�ncia de todas as salas $(\sum_{i=0}^n U_i)$ , no qual o melhor resultado consiste em manter o menor valor da somat�ria dos graus de urg�ncia; mostrando que o algoritmo ou estrat�gia levou o rob� �s salas com maior efici�ncia. O grau de urg�ncia $U$ � calculado multiplicando-se a prioridade relatica da sala $P/P_t$ pelo tempo decorrido desde a �ltima visita $t$ $(U_i = P_i/P_t \times t_i)$. A casa visita o $t$ � zerado.

\section{Metodologia}
\label{metodologia}
%#######################################################################################################
%Para a solu��o do problema descrito anteriormente, sup�e-se que o rob� tenha uma descri��o do ambiente em que atua (mapa). 
%Os sensores que foram utilizados no projeto foram LASER e od�metro.
%\section{Controle do Rob�}
%\label{controle_do_robo}
%O controle do rob� ser� desenvolvido utilizando-se a biblioteca Player/Stage \cite{Gerkey2003}, a qual permite que sejam realizadas simula��es antes que os algoritmos %sejam testados com o rob� real. Essas simula��es s�o importantes para o aperfei�oamento dos par�metros como dist�ncias, velocidades, etc, antes do teste no Pioneer.
%O m�todo utilizado foi o VFH, por j� vir implementado no Player e em testes se mostrou suficiente para o projeto.
%#######################################################################################################
Para a solu��o do problema descrito, o rob� tem uma descri��o completa do ambiente em que atua (mapa). Foram utilizados somente dois sensores: Od�metro para localizar o rob� no mapa e o LASER para o desvio de obst�culos. O controle do rob� foi desenvolvido utilizando a biblioteca Player/Stage. Para desvio de obst�culos utilizou-se a t�cnica VFH que j� vem implementada no Player e se mostrou suficiente para o projeto.

Cada ambiente tem um mapa topol�gico. % que ser� gerado utilizando o diagrama de Voronoi \ref{voronoi}). 
O rob� deve utilizar esse mapa para se locomover de uma sala para outra no ambiente. Para os algoritmos e estrat�gias determinarem a seq��ncia de salas a serem visitadas, � considerado um grafo completo com todas as salas, pois para determinar o melhor caminho entre uma sala e outra ser� utilizado o algoritmo de Dijkstra no mapa topol�gico.

Como um dos crit�rios de avalia��o est� relacionado ao tempo que o rob� fica sem visitar as salas � poss�vel deduzir que a solu��o seja um ciclo.

Considera-se:
\begin{itemize}
	\item Um ambiente com $S$ salas;
	\item Um ciclo hamiltoniano $C$ qualquer;
	\item $C_i$ � a i-�sima sala visitada no ciclo;
	\item Uma velocidade constante do rob� (tanto linear quanto angular);
	\item $\Delta t_i$ o tempo para sair da sala $i$ e chegar na sala $i+1$.
\end{itemize}

\pagebreak
Como o tempo de viagem entre as mesmas salas � constante todos os $\Delta t_i$ s�o constantes, portanto o tempo total do ciclo $T$ � constante $(\sum_{i=1}^S \Delta t_i = cte)$. Se o tempo do ciclo � constante, o tempo que o rob� demora para revisitar cada sala � constante igual a $T$. Mas isso n�o � interessante para solu��o do problema pois o grau de urg�ncia de cada sala � diferente, ent�o se uma sala tem uma prioridade muito alta, seu grau de urg�ncia vai ser muito alto at� o rob� revisit�-la. 

Para resolver esse problema, basta fazer com que o rob� revisite essa(s) sala(s) mais de uma vez no ciclo. Ent�o supondo um ciclo de tamanho $n$ sendo $n \geq S$ o tempo total do ciclo continua constante $(\sum_{i=1}^n \Delta t_i = cte)$. Portanto a solu��o do problema consiste em encontrar esse ciclo.

\section{Solu��es}
\label{solucoes}

Foram definidos dois tipos de solu��es: uma solu��o \textbf{\textit{offline}}, ou seja, a solu��o � calculada em um computador e depois informada ao rob� qual seq��ncia de sala ele deve seguir, essa seq��ncia n�o � alterada. E a outra � uma solu��o \textbf{tempo real}, ou seja, o rob� define para qual sala deve ir durante a execu��o do algoritmo, baseando suas decis�es no que est� acontecendo no momento. A solu��o \textit{\textbf{offline}} � dividida em duas partes: \textbf{Gerador}, que cria poss�veis seq��ncias de salas �timas, e o \textbf{Avaliador} que analisa e computa uma nota para as seq��ncias de salas criadas pelo \textbf{Gerador}

\subsection{\textit{Offline}}
\label{offline}
Esse m�todo consiste em explorar as diversas poss�veis combina��es de seq��ncias de salas para encontrar a seq��ncia �tima antes de informar ao rob� qual seq��ncia de salas deve seguir. Um programa chamado \texttt{gerador}, seguindo uma determinada heur�stica, gera as poss�veis seq��ncias de salas, essas seq��ncias s�o fornecidas ao programa \texttt{avaliador} que, baseado no crit�rio de maior Grau de Urg�ncia Total, analisa a seq��ncia e retorna uma avalia��o ao \texttt{gerador} que tomar� a decis�o de: descartar a seq��ncia, preservar a seq��ncia para gerar futuras seq��ncias ou guardar a seq��ncia como poss�vel seq��ncia �tima.


\subsubsection{Gerador}
\label{gerador}

O programa chamado \texttt{gerador} � utilizado para gerar os candidatos � seq��ncia de salas �tima do mapa analisado. Esses candidatos s�o avaliados pelo programa \texttt{avaliador} e de acordo com a avalia��o o \texttt{gerador} ir� guardar a seq��ncia como poss�vel �tima, descartar ou continuar utilizando essa seq��ncia para gerar novas seq��ncias. 

%\subsubsection{Funcionamento}
%\paragraph{Funcionamento}

O programa \texttt{gerador} utiliza uma classe (tipo de vari�vel) chamada $Agente$ contendo os seguintes atributos:

\begin{itemize}
	\item \texttt{vertice}: V�rtice no qual o agente se encontra no momento;
	\item \texttt{caminho}: Vetor de salas que guarda a seq��ncia de salas que o agente percorreu at� chegar no v�rtice atual;
	\item \texttt{avaliacao}: Avalia��o da seq��ncia de salas do agente;
	\item \texttt{tempo}: Tempo em segundos que o rob� levou para percorrer a seq��ncia de salas at� o v�rtice atual.
\end{itemize}

O \texttt{gerador} recebe apenas um par�metro como entrada: o nome do mapa que quer achar a seq��ncia �tima. O programa inicia criando uma seq��ncia percorrendo as salas na ordem num�rica (Sala 1, sala 2..) e classificada como poss�vel �tima. A seguir o programa cria um $Agente$ em cada sala do mapa para explorar as possibilidades do rob� come�ar em cada sala, esses $agentes$ s�o organizados em uma fila de candidatos.

O la�o principal do programa consiste em retirar um $agente$ da fila e para cada sala $i$ � criado um novo $agente$, simulando que o $agente$ retirado da fila navegou at� a sala $i$. Em cada novo $agente$ � atualizada a seq��ncia de salas visitadas e o agente � reavaliado. Se a avalia��o for um valor negativo o $agente$ � inserido na fila de $agentes$. Se o valor retornado for 0 (zero) o $agente$ � descartado. Caso o avaliador retorne um valor positivo e esse valor � menor que a avalia��o da atual seq��ncia �tima, o novo $agente$ contendo a seq��ncia melhor � guardado como poss�vel seq��ncia �tima. Como o algoritmo somente insere na fila os $agentes$ que podem gerar seq��ncias �timas, o crit�rio de parada � quando n�o existe mais $agentes$ na fila, o programa ent�o retorna o $agente$ com a seq��ncia �tima.

A Figura \ref{fig:fluxograma_gerador} mostra o fluxograma do programa \texttt{gerador} e o algoritmo \ref{alg:gerador} cont�m seu pseudoc�digo.
\begin{figure}[htbp]
	\centering
		\includegraphics[width=1\textwidth]{../../imagens/fluxograma_gerador.png}
	\caption{Fluxograma do \texttt{gerador}}
	\label{fig:fluxograma_gerador}
\end{figure}
%\subsection{Algoritmo}

\begin{algorithm}[htbp]
\caption{\texttt{gerador}}
\label{alg:gerador}
\begin{algorithmic}[1]
\REQUIRE  mapa
\ENSURE $melhor\_agente$
	\STATE $agente\_otimo.vertice \leftarrow 1$
	\STATE $agente\_otimo.caminho \leftarrow 1, 2, 3...$
	\STATE $agente\_otimo.avaliacao \leftarrow avaliar(agente\_otimo)$
	\FORALL {sala $s$ do mapa}
		\STATE Inicia $agente$ na sala $s$
		\STATE Adiciona $agente$ na $fila$
	\ENDFOR

	\REPEAT
		\STATE Retira o $agente$ do topo da fila
		\FORALL {sala $s$ do mapa}
			\STATE $novo\_agente \leftarrow agente$
			\STATE Atualiza o v�rtice do $agente$ com a sala $s$
			\STATE Adiciona a sala $s$ ao caminho do $novo\_agente$
			\STATE Avalia o $novo\_agente$ como o caminho atual
			
			\IF {A avalia��o for negativa}
				\STATE \COMMENT {\textit{$novo\_agente$ n�o visitou todas as salas ou n�o � um loop}}
				\STATE Adiciona o $novo\_agente$ no fim da $fila$
			\ENDIF
			\IF {A avalia��o for positiva}
				\STATE \COMMENT {\textit{$novo\_agente$ � melhor que o �timo atual}}
%				\STATE Coloca o $agente\_otimo$ no fim da fila \COMMENT {\textit{$agente\_otimo$ pode gerar um caminho �timo}}
				\STATE $agente\_otimo \leftarrow novo\_agente$ \COMMENT {\textit{Atualiza o $agente\_otimo$ com o $novo\_agente$}}
			\ENDIF
		\ENDFOR
	\UNTIL N�o exista $agente$ na fila
	\RETURN $agente\_otimo$

\end{algorithmic}
\end{algorithm}

\clearpage

A seguir, uma explica��o mais detalhada de como o \texttt{gerador} cria as poss�veis seq��ncias de salas para serem avaliadas. Ser� usado como exemplo um ambiente com quatro salas. Independente do mapa topol�gico, o gerador trabalha utilizando um grafo completo contendo somente as salas (Figura \ref{fig:gerador1}).

\begin{figure}[hb]
	\centering
		\includegraphics[width=1.00\textwidth]{../../imagens/gerador1.png}
	\caption{Grafo completo das salas.}
	\label{fig:gerador1}
\end{figure}

O \texttt{gerador} ent�o cria um \textit{agente} em cada sala como mostra a Figura \ref{fig:gerador2}.

\begin{figure}[hb]
	\centering
		\includegraphics[width=1.00\textwidth]{../../imagens/gerador2.png}
	\caption{Um agente em cada sala, e a fila de \textit{agentes}.}
	\label{fig:gerador2}
\end{figure}

Seguindo o la�o principal, o \textit{gerador} remove o primeiro \textit{agente} da fila, no caso � o \textit{agente} \textbf{a} (Figura \ref{fig:gerador3}).

\begin{figure}[hb]
	\centering
		\includegraphics[width=1.00\textwidth]{../../imagens/gerador3.png}
	\caption{Removendo o \textit{agente} \textbf{a}.}
	\label{fig:gerador3}
\end{figure}

Em seguida o \texttt{gerador} cria uma c�pia do \textit{agente} \textbf{a} para cada sala, e simula que esses \textit{agentes} navegaram at� sua sala correspondente (Figura \ref{fig:gerador4}). Atualizando o caminho que esse agente percorreu para chegar a essa sala e a avalia��o desse caminho.

\begin{figure}[hb]
	\centering
		\includegraphics[width=1.00\textwidth]{../../imagens/gerador4.png}
	\caption{Inserindo os \textit{agentes} \textbf{e}, \textbf{f} e \textbf{g}.}
	\label{fig:gerador4}
\end{figure}
%
Ap�s feito isso o \texttt{gerador} recome�a o ciclo, removendo o primeiro \textit{agente} da fila (no caso \textit{agente} \textbf{b}), criando outros \textit{agentes} e inserindo-os na fila de acordo com a avalia��o adquirida (\textit{agentes} \textbf{h}, \textbf{i} e \textbf{j}) como mostra a Figura \ref{fig:gerador5}.

\begin{figure}[hb]
	\centering
		\includegraphics[width=1.00\textwidth]{../../imagens/gerador5.png}
	\caption{Removendo o \textit{agente} \textbf{b} e inserindo os \textit{agentes} \textbf{h}, \textbf{i} e \textbf{j}.}
	\label{fig:gerador5}
\end{figure}

O \texttt{gerador} permanece nesse ciclo at� que n�o exista mais \textit{agente} na fila.

\subsubsection{Avaliador}
\label{avaliador}

O programa chamado \texttt{avaliador} � utilizado para avaliar os candidatos a seq��ncia de salas �tima do mapa analisado gerado pelo programa \texttt{gerador} (Se��o \ref{gerador}) e retorna para o \texttt{gerador} um valor num�rico. Baseado nesse valor o \texttt{gerador} ir� guardar a seq��ncia como poss�vel �tima, descartar ou continuar utilizando essa seq��ncia para gerar novas seq��ncia.

%\subsubsection{Funcionamento}
%\paragraph{Funcionamento}

O \texttt{avaliador} recebe tr�s par�metros como entrada:
\begin{itemize}
	\item O nome do mapa onde o caminho ser� avaliado;
	\item Um vetor de v�rtices, representando a seq��ncia de salas a ser avaliada;
	\item Um valor opcional de limite para a avalia��o.
\end{itemize}

Com esses par�metros o \texttt{avaliador} inicia a simula��o do rob� navegando pelo mapa. A velocidade linear, que determina a velocidade com que o rob� anda para frente ou para tr�s, � definida pela constante \texttt{SIMULACAO\_VEL} como 1 m/s. A velocidade angular, que determina a velocidade com que o rob� vira, � definida pela constante \texttt{SIMULACAO\_ROT} como 0,5 rad/s. O tempo que o rob� demora para visitar uma sala � definido pela constante \texttt{VISITAR\_SALA} como 5 s. Todos esses valores foram definidos arbritrariamente.

Durante a simula��o, a cada visita de sala o \texttt{avaliador} mede o Grau de Urg�ncia Total e salva o maior valor, esse valor vai ser a avalia��o do caminho neste mapa. Ap�s a simula��o o \texttt{avaliador} verifica se a avalia��o do caminho for maior que o limite fornecido o valor 0 (zero) � retornado, se nenhum limite for fornecido o \texttt{avaliador} utiliza o valor do maior inteiro do compilador. Se o caminho n�o visita todas as salas o \texttt{avaliador} retorna a avalia��o com valor negativo, se todas as salas foram visitadas o \texttt{avaliador} verifica se o caminho � um \textit{loop}, ou seja, se o caminho come�a e termina no mesmo v�rtice, caso contr�rio ele tamb�m retorna a avalia��o em valor negativo.  Se o caminho for um \textit{loop}, visita todas as salas e o valor da avalia��o for menor que o limite fornecido (ou menor que o maior inteiro do compilador), o valor da avalia��o, maior Grau de Urg�ncia Total em todo o percurso, � retornado pelo \texttt{avaliador}.

Para crit�rio de parada o \texttt{avaliador} compara o estado das salas, ou seja, se para cada sala, o Grau de Urg�ncia e o n�mero de visitas � o mesmo ao t�rmino do caminho.

A Figura \ref{fig:fluxograma_avaliador} mostra o fluxograma do programa \texttt{avaliador} e o algoritmo \ref{alg:avaliador} cont�m seu pseudoc�digo.

\begin{figure}[htbp]
	\centering
		\includegraphics[width=1.00\textwidth]{../../imagens/fluxograma_avaliador.png}
	\caption{Fluxograma do \texttt{avaliador}}
	\label{fig:fluxograma_avaliador}
\end{figure}


%\subsection{Algoritmo}

\begin{algorithm}[htbp]
\caption{\texttt{avaliador}}
\label{alg:avaliador}
\begin{algorithmic}[1]
\REQUIRE mapa, caminho, limite = \texttt{INT\_MAX} 
\ENSURE A avalia��o do $caminho$ no $mapa$
	\STATE carregar ($mapa$)
	\STATE $aval \leftarrow 0$
	\REPEAT
		\STATE $salas\_anterior \leftarrow salas\_atual$
		\FORALL {v�rtice $v$ em $caminho$}
			\STATE Calcula o tempo necess�rio para o rob� chegar ao v�rtice $v$
			\STATE Soma a esse tempo o tempo para visitar a sala do v�rtice $v$
			\STATE Atualiza os Graus de Urg�ncia das salas com o tempo calculado
			\IF {Grau de Urg�ncia atual > $maior$}
				\STATE $maior \leftarrow$ Grau de Urg�ncia atual
			\ENDIF
			\STATE Visita a sala do v�rtice $v$
		\ENDFOR

		\IF {$aval$ > $limite$}
			\RETURN 0
		\ENDIF

		\IF {Caminho n�o visita todas as salas do mapa}
			\RETURN $-aval$
		\ENDIF

		\IF {Primeiro v�rtice do caminho diferente do �ltimo v�rtice do caminho} 
		\STATE \COMMENT{\textit{Ou seja, n�o � considerado um loop}}
			\RETURN $-aval$
		\ENDIF
	
	\UNTIL $salas\_anterior$ = $salas\_atual$

	\RETURN $aval$
\end{algorithmic}
\end{algorithm}

\clearpage
%\pagebreak

\subsection{Tempo Real}
\label{tempo_real}
A solu��o em tempo real inicia as prioridades de todas as salas em zero, cada sala recebe um valor correspondente � chance dessa sala gerar uma emerg�ncia. Emerg�ncia � quando a sala solicita que o rob� v� visit�-la, como uma lixeira cheia ou a sala n�o ser visitada por um tempo m�nimo determinado por exemplo. A cada emerg�ncia gerada a prioridade da sala � acrescida de uma unidade. O rob� segue o paradigma de ir � sala de maior grau de urg�ncia, visita a sala e procura a pr�xima sala de maior grau de urg�ncia.

A chance de emerg�ncia que cada sala recebe � proporcional �s prioridades, para fazer o c�lculo basta dividir a prioridade da sala pela soma das prioridades do mapa.

\begin{equation}
	p_s = \frac{P_s}{\sum_{i=1}^{TotalSalas} P_i}
	\label{eq:calc-chances}	
\end{equation}

Isso faz com que a soma das chances seja igual a um. Por exemplo, se em um mapa de quatro salas as prioridades forem (Tabela \ref{tab:ExemploDePrioridades}):

\begin{table*}[hb]
	\caption{Exemplo de prioridades}
	\centering
		\begin{tabular}
			{|c|c|}
			\hline
			Sala & Prioridade \\
			\hline
			1&5\\
			2&1\\
			3&4\\
			4&5\\					
			\hline
		\end{tabular}
	\label{tab:ExemploDePrioridades}
\end{table*}

Suas chances de gerar uma emerg�ncia ficaram da seguinte forma (Tabela \ref{tab:ExemploDeChancesDeEmerg�ncia}):

\begin{table*}[hb]
	\caption{Exemplo de chances de emerg�ncia}
	\centering
		\begin{tabular}
			{|c|c|}
			\hline
			Sala & Chance \\
			\hline
			1&0,333\\
			2&0,067\\
			3&0,267\\
			4&0,333\\					
			\hline
		\end{tabular}
	\label{tab:ExemploDeChancesDeEmerg�ncia}
\end{table*}

\section{Resultados}
\label{resultados}

Para a solu��o em Tempo Real, por se tratar de uma solu��o baseada em n�meros aleat�rios, foram executados 100 repeti��es em cada mapa.

Para os testes foram criados sete mapas, cada qual com uma peculiaridade diferente como mostra a tabela \ref{tab:mapas_para_testes}
\begin{table}[ht]
	\centering
	\caption{Mapas para testes}
		\begin{tabular}{|c|c|c|}
			\hline
			Designa��o & N\# de Salas & Prioridades \\
			\hline
			A			& 4 & Iguais		 \\
			B	& 4 & Diferentes \\
			C						& 5 & Diferentes \\
			D& 5 & Diferentes \\
			E			& 6 & Iguais		 \\
			F& 6 & Diferentes \\
			G	& 8 & Diferentes \\
			\hline
		\end{tabular}
	\label{tab:mapas_para_testes}
\end{table}

Na seq��ncia s�o mostrados os resultados em cada mapa:

\begin{itemize}
	\item Mapa A: Com quatro salas de prioridades iguais (Tabela \ref{tab:P_quatro}) e liga��o entre todas as salas (Figura \ref{fig:grafo_quatro}):

\begin{figure}[htb]
	\centering
		\includegraphics[width=0.25\textwidth]{../../imagens/mapa_quatro.png}
	\caption{Grafo do Mapa A e B.}
	\label{fig:grafo_quatro}
\end{figure}

\begin{table}[hb]
	\centering
	\caption{Prioridades do Mapa A.}
 	\begin{tabular}{|c|c|c|}
		\hline
		Sala & Prioridade\\
		\hline
		1& 1 \\
		2& 1 \\
		3& 1 \\
		4& 1 \\
		\hline
  \end{tabular}
  \label{tab:P_quatro}
\end{table}

A seq��ncia de salas a serem visitadas gerada pelo \texttt{gerador} foi: 1 2 4 3.

As prioridades relativas finais da solu��o em Tempo Real foram pr�ximas �s prioridades relativas iniciais do mapa, como mostra a Tabela \ref{tab:Pr_quatro}.		
 		
\begin{table}[!ht]		
	\centering	
	\caption{Prioridades relativas do Mapa A.}	
	\begin{tabular}{|c|c|c|}	
		\hline
		Sala & \textit{Offline} & Tempo Real \\
		\hline
		1 & 0,25 & 0,24989 \\
		2 & 0,25 & 0,24860 \\
		3 & 0,25 & 0,25224 \\
		4 & 0,25 & 0,24927 \\
		\hline
	\end{tabular}	
	\label{tab:Pr_quatro}	
\end{table}		
 		
A Tabela \ref{tab:Fr_quatro} mostra que as freq��ncias relativas tanto da solu��o \textit{Offline} quanto da solu��o em Tempo Real foram pr�ximas �s prioridades relativas iniciais do mapa. A tabela mostra tamb�m que a solu��o \textit{Offline} foi melhor que a solu��o em Tempo Real atrav�s da diferen�a quadr�tica.
 		
\begin{table}[ht]
	\centering	
	\caption{Freq��ncias Relativas do Mapa A.}	
	\begin{tabular}{cc|c|c|c|c|}	
		\cline{3-6}
		& & \multicolumn{2}{c|}{\textit{Offline}}& \multicolumn{2}{c|}{Tempo Real}\\
		\hline
		\multicolumn{1}{|c|}{Sala} & Prioridade Rel. & Freq. & Freq. Rel. & Freq. & Freq. Rel. \\
		\hline
		\multicolumn{1}{|c|}{1} & 0,25 & 901 & 0,25021 & 771,31 & 0,25114 \\
		\multicolumn{1}{|c|}{2} & 0,25 & 900 & 0,24993 & 775,88 & 0,25263 \\
		\multicolumn{1}{|c|}{3} & 0,25 & 900 & 0,24993 & 764,04 & 0,24877 \\
		\multicolumn{1}{|c|}{4} & 0,25 & 900 & 0,24993 & 760,02 & 0,24746 \\
		\hline
		\multicolumn{2}{|c|}{Diferen�a Quadr�tica} & \multicolumn{2}{c|}{0,00000} & \multicolumn{2}{c|}{0,00002} \\
		\hline
	\end{tabular}	
	\label{tab:Fr_quatro}	
\end{table}			

O gr�fico (Figura \ref{fig:grafico_quatro}) mostra que a solu��o em Tempo Real depois de estabilizar teve um desempenho inferior � solu��o \textit{Offline}.
\begin{figure}[!hb]
	\centering
		\includegraphics[width=1\textwidth]{../../../resultados/graficos/quatro.png}
	\caption{Gr�fico do Mapa A.}
	\label{fig:grafico_quatro}
\end{figure}

%\newpage
\clearpage

Como o Mapa A � relativamente pequeno (dois metros quadrados) e a seq��ncia gerada pela solu��o \textit{Offline} � visitar as salas no sentido anti-hor�rio, o rob� leva exatamente o mesmo tempo para ir de uma sala � outra. Como as prioridades iniciais do mapa s�o todas iguais e para gerar o gr�fico foram tiradas medidas (Grau de Urg�ncia Total) de dez em dez segundos, coincidiu de o rob� a cada dez segundos estar no mesmo lugar no mapa, por isso o gr�fico do Mapa A para a solu��o \textit{Offline} � uma linha. 
%Fato que n�o acontece no gr�fico do Mapa B (Figura \ref{fig:grafico_quatro_diff}), pois as prioridades n�o s�o todas iguais.
 
	\item Mapa B: Com quatro salas com prioridades diferentes (Tabela \ref{tab:P_quatro_diff}) e liga��o entre todas as salas (Figura \ref{fig:grafo_quatro}):
	
\begin{table}[hb]
	\centering
	\caption{Prioridades do Mapa B.}
 	\begin{tabular}{|c|c|c|}
		\hline
		Sala & Prioridade\\
		\hline
			1& 1 \\
			2& 1 \\
			3& 5 \\
			4& 5 \\
		\hline
  \end{tabular}
  \label{tab:P_quatro_diff}
\end{table}

A seq��ncia de salas a serem visitadas gerada pelo \texttt{gerador} foi: 1 3 4 2 4 3.

As prioridades relativas finais da solu��o em Tempo Real foram pr�ximas �s prioridades relativas iniciais do mapa, como mostra a Tabela \ref{tab:Pr_quatro_diff}.		
 		
\begin{table}[hb]		
	\centering	
	\caption{Prioridades relativas do Mapa B.}	
	\begin{tabular}{|c|c|c|}	
		\hline
		Sala & \textit{Offline} & Tempo Real \\
		\hline
		1 & 0,08333 & 0,08470 \\
		2 & 0,08333 & 0,08331 \\
		3 & 0,41667 & 0,41141 \\
		4 & 0,41667 & 0,42058 \\
		\hline
	\end{tabular}	
	\label{tab:Pr_quatro_diff}	
\end{table}		
 		
A Tabela \ref{tab:Fr_quatro_diff} mostra as freq��ncias relativas das solu��es \textit{Offline} e em Tempo Real. Nota-se que as freq��ncias relativas das salas 1 e 2 da solu��o \textit{Offline} s�o aproximadamente o dobro das suas respectivas prioridades relativas iniciais do mapa, por�m, em n�meros absolutos, � uma diferen�a de aproximadamente 0,08. E as freq��ncias relativas da solu��o em Tempo Real s�o pr�ximas �s suas respectivas prioridades relativas iniciais do mapa. A tabela mostra tamb�m que a solu��o em Tempo Real foi melhor que a solu��o \textit{Offline} atrav�s da diferen�a quadr�tica.

		
\begin{table}		
	\centering	
	\caption{Freq��ncias Relativas do Mapa B.}	
	\begin{tabular}{cc|c|c|c|c|}	
		\cline{3-6}
		& & \multicolumn{2}{c|}{\textit{Offline}}& \multicolumn{2}{c|}{Tempo Real}\\
		\hline
		\multicolumn{1}{|c|}{Sala} & Prioridade Rel. & Freq. & Freq. Rel. & Freq. & Freq. Rel. \\
		\hline
		\multicolumn{1}{|c|}{1} & 0,08333 & 546 & 0,16677 & 300,37 & 0,10130 \\
		\multicolumn{1}{|c|}{2} & 0,08333 & 546 & 0,16677 & 296,35 & 0,09994 \\
		\multicolumn{1}{|c|}{3} & 0,41667 & 1091 & 0,33323 & 1192,15 & 0,40206 \\
		\multicolumn{1}{|c|}{4} & 0,41667 & 1091 & 0,33323 & 1176,27 & 0,39670 \\
		\hline
		\multicolumn{2}{|c|}{Diferen�a Quadr�tica} & \multicolumn{2}{c|}{0,02785} & \multicolumn{2}{c|}{0,00121} \\
		\hline
	\end{tabular}	
	\label{tab:Fr_quatro_diff}	
\end{table}		

O gr�fico do Mapa B (Figura \ref{fig:grafico_quatro_diff}), diferentemente do gr�fico do Mapa A (Figura \ref{fig:grafico_quatro}), ilustra que a solu��o \textit{Offline} manteve o Grau de Urg�ncia Total dentro de uma faixa, entre dez e trinta aproximadamente, mas seu desempenho tamb�m foi superior � solu��o em Tempo Real.
\begin{figure}[htb]
	\centering
		\includegraphics[width=1\textwidth]{../../../resultados/graficos/quatro_diff.png}
	\caption{Gr�fico do Mapa B.}
	\label{fig:grafico_quatro_diff}
\end{figure}


\item Mapa C: Com cinco salas que formam um X (Prioridades na Tabela \ref{tab:P_x}) com liga��es entre as salas das pontas (Figura \ref{fig:grafo_x}):
	
\begin{figure}[!hb]
	\centering
		\includegraphics[width=0.29\textwidth]{../../imagens/mapa_x.png}
	\caption{Grafo do Mapa C.}
	\label{fig:grafo_x}
\end{figure}

\clearpage

\begin{table}[hb]
	\centering
	\caption{Prioridades do Mapa C e D.}
 	\begin{tabular}{|c|c|c|}
		\hline
		Sala & Prioridade\\
		\hline
			1& 1 \\
			2& 5 \\
			3& 5 \\
			4& 5 \\
			5& 5 \\
		\hline
  \end{tabular}
  \label{tab:P_x}
\end{table}

A seq��ncia de salas a serem visitadas gerada pelo \texttt{gerador} foi: 1 2 4 5 3 2 4 5 3.

As prioridades relativas finais da solu��o em Tempo Real foram pr�ximas �s prioridades relativas iniciais do mapa, como mostra a Tabela \ref{tab:Pr_x}.		
 		
\begin{table}[hb]		
	\centering	
	\caption{Prioridades relativas do Mapa C.}	
	\begin{tabular}{|c|c|c|}	
		\hline
		Sala & \textit{Offline} & Tempo Real \\
		\hline
		1 & 0,04762 & 0,04983 \\
		2 & 0,23810 & 0,24044 \\
		3 & 0,23810 & 0,23870 \\
		4 & 0,23810 & 0,23501 \\
		5 & 0,23810 & 0,23602 \\
		\hline
	\end{tabular}	
	\label{tab:Pr_x}	
\end{table}		
 		
A Tabela \ref{tab:Fr_x} mostra que freq��ncia relativa da sala 1 na solu��o \textit{Offline} foi maior que o dobro da sua prioridade relativa inicial do mapa, entretanto, a diferen�a num�rica absoluta � aproximadamente 0,06. As freq��ncias relativas das salas restantes da solu��o \textit{Offline} e as freq��ncias relativas de todas as salas da solu��o em Tempo Real foram pr�ximas �s prioridades relativas iniciais do mapa. A tabela \ref{tab:Fr_x} mostra tamb�m que a solu��o Tempo Real foi melhor que a solu��o \textit{Offline} atrav�s da diferen�a quadr�tica.

\begin{table}[!hb]
	\centering	
	\caption{Freq��ncias Relativas do Mapa C.}	
	\begin{tabular}{cc|c|c|c|c|}	
		\cline{3-6}
		& & \multicolumn{2}{c|}{\textit{Offline}}& \multicolumn{2}{c|}{Tempo Real}\\
		\hline
		\multicolumn{1}{|c|}{Sala} & Prioridade Rel. & Freq. & Freq. Rel. & Freq. & Freq. Rel. \\
		\hline
		\multicolumn{1}{|c|}{1} & 0,04762 & 328 & 0,11134 & 151,35 & 0,05670 \\
		\multicolumn{1}{|c|}{2} & 0,23810 & 655 & 0,22234 & 633,44 & 0,23729 \\
		\multicolumn{1}{|c|}{3} & 0,23810 & 654 & 0,22200 & 635,52 & 0,23807 \\
		\multicolumn{1}{|c|}{4} & 0,23810 & 655 & 0,22234 & 619,58 & 0,23210 \\
		\multicolumn{1}{|c|}{5} & 0,23810 & 654 & 0,22200 & 629,54 & 0,23583 \\
		\hline
		\multicolumn{2}{|c|}{Diferen�a Quadr�tica} & \multicolumn{2}{c|}{0,00508} & \multicolumn{2}{c|}{0,00012} \\
		\hline
\end{tabular}	
	\label{tab:Fr_x}	
\end{table}		

O gr�fico (Figura \ref{fig:grafico_x}) mostra que a solu��o em Tempo Real depois de estabilizar teve um desempenho inferior � solu��o \textit{Offline}.
\clearpage
\begin{figure}[!htb]
	\centering
		\includegraphics[width=1\textwidth]{../../../resultados/graficos/x.png}
	\caption{Gr�fico do Mapa C.}
	\label{fig:grafico_x}
\end{figure}



	\item Mapa D: Com cinco salas em formato de X (Prioridades iguais ao mapa anterior indicados na Tabela \ref{tab:P_x}) com somente a sala do meio ligando as outras (Figura \ref{fig:grafo_x_incompleto}):
	
\begin{figure}[htb]
	\centering
		\includegraphics[width=0.3\textwidth]{../../imagens/mapa_x_incompleto.png}
	\caption{Grafo do Mapa D.}
	\label{fig:grafo_x_incompleto}
\end{figure}

A seq��ncia de salas a serem visitadas gerada pelo \texttt{gerador} foi: 1 2 3 4 5.

As prioridades relativas finais da solu��o em Tempo Real foram pr�ximas �s prioridades relativas iniciais do mapa, como mostra a Tabela \ref{tab:Pr_x_incompleto}.		
\clearpage 		
\begin{table}[!hbt]		
	\centering	
	\caption{Prioridades relativas do Mapa D.}	
	\begin{tabular}{|c|c|c|}	
		\hline
		Sala & \textit{Offline} & Tempo Real \\
		\hline
		1 & 0,04762 & 0,04598 \\
		2 & 0,23810 & 0,23794 \\
		3 & 0,23810 & 0,24530 \\
		4 & 0,23810 & 0,23984 \\
		5 & 0,23810 & 0,23094 \\
		\hline
	\end{tabular}	
	\label{tab:Pr_x_incompleto}	
\end{table}		
 		
A Tabela \ref{tab:Fr_x_incompleto} mostra que as freq��ncias relativas da solu��o \textit{Offline} foram aproximadamente todas iguais, isso se deve ao fato de que a seq��ncia de salas a serem visitadas gerada pelo \texttt{gerador} manda o rob� visitar cada sala apenas uma �nica vez dentro do \textit{loop}. Na mesma tabela pode-se ver tamb�m que as freq��ncias relativas da solu��o em Tempo Real foram pr�ximas �s prioridades relativas iniciais do mapa. A tabela mostra tamb�m que a solu��o Tempo Real foi melhor que a solu��o \textit{Offline} atrav�s da diferen�a quadr�tica.
 		
\begin{table}[!hbt]
	\centering	
	\caption{Freq��ncias Relativas do Mapa D.}	
	\begin{tabular}{cc|c|c|c|c|}	
		\cline{3-6}
		& & \multicolumn{2}{c|}{\textit{Offline}}& \multicolumn{2}{c|}{Tempo Real}\\
		\hline
		\multicolumn{1}{|c|}{Sala} & Prioridade Rel. & Freq. & Freq. Rel. & Freq. & Freq. Rel. \\
		\hline
		\multicolumn{1}{|c|}{1} & 0,04762 & 456 & 0,20009 & 107,18 & 0,05481 \\
		\multicolumn{1}{|c|}{2} & 0,23810 & 456 & 0,20009 & 467,55 & 0,23908 \\
		\multicolumn{1}{|c|}{3} & 0,23810 & 456 & 0,20009 & 462,17 & 0,23633 \\
		\multicolumn{1}{|c|}{4} & 0,23810 & 456 & 0,20009 & 455,16 & 0,23275 \\
		\multicolumn{1}{|c|}{5} & 0,23810 & 455 & 0,19965 & 463,53 & 0,23703 \\
		\hline
		\multicolumn{2}{|c|}{Diferen�a Quadr�tica} & \multicolumn{2}{c|}{0,02906} & \multicolumn{2}{c|}{0,00009} \\
		\hline
	\end{tabular}	
	\label{tab:Fr_x_incompleto}	
\end{table}		

O gr�fico (Figura \ref{fig:grafico_x_incompleto}) mostra que a solu��o em Tempo Real depois de estabilizar teve um desempenho inferior � solu��o \textit{Offline}.
\begin{figure}[htb]
	\centering
		\includegraphics[width=1\textwidth]{../../../resultados/graficos/x_incompleto.png}
	\caption{Gr�fico do Mapa D.}
	\label{fig:grafico_x_incompleto}
\end{figure}


	\item Mapa E: Com seis salas de prioridades iguais (Tabela \ref{tab:P_espinha}) em formato de espinha de peixe (Figura \ref{fig:grafo_espinha}):

\begin{figure}[!htb]
	\centering
		\includegraphics[width=0.3\textwidth]{../../imagens/mapa_espinha.png}
	\caption{Grafo do Mapa E e F.}
	\label{fig:grafo_espinha}
\end{figure}

\begin{table}[!htb]
	\centering
	\caption{Prioridades do Mapa E.}
 	\begin{tabular}{|c|c|c|}
		\hline
		Sala & Prioridade\\
		\hline
		1& 2 \\
		2& 2 \\
		3& 2 \\
		5& 2 \\
		6& 2 \\
		7& 2 \\
		\hline
  \end{tabular}
  \label{tab:P_espinha}
\end{table}


A seq��ncia de salas a serem visitadas gerada pelo \texttt{gerador} foi: 1 5 2 6 3 7.

As prioridades relativas finais da solu��o em Tempo Real foram pr�ximas �s prioridades relativas iniciais do mapa, como mostra a Tabela \ref{tab:Pr_espinha}.		
 
\begin{table}[!hbt]		
	\centering	
	\caption{Prioridades relativas do Mapa E.}	
	\begin{tabular}{|c|c|c|}	
		\hline
		Sala & \textit{Offline} & Tempo Real \\
		\hline
		1 & 0,16667 & 0,16990 \\
		2 & 0,16667 & 0,16613 \\
		3 & 0,16667 & 0,16571 \\
		5 & 0,16667 & 0,16339 \\
		6 & 0,16667 & 0,16720 \\
		7 & 0,16667 & 0,16768 \\
		\hline
	\end{tabular}	
	\label{tab:Pr_espinha}	
\end{table}		
 		
\clearpage

A Tabela \ref{tab:Fr_espinha} mostra que as freq��ncias relativas tanto da solu��o \textit{Offline} quanto a em Tempo Real foram pr�ximas �s prioridades relativas iniciais do mapa.	A Tabela \ref{tab:Fr_espinha} mostra tamb�m que a solu��o \textit{Offline} foi melhor que a solu��o Tempo Real atrav�s da diferen�a quadr�tica.	
 		
\begin{table}[!hbt]
	\centering	
	\caption{Freq��ncias Relativas do Mapa E.}	
	\begin{tabular}{cc|c|c|c|c|}	
		\cline{3-6}
		& & \multicolumn{2}{c|}{\textit{Offline}}& \multicolumn{2}{c|}{Tempo Real}\\
		\hline
		\multicolumn{1}{|c|}{Sala} & Prioridade Rel. & Freq. & Freq. Rel. & Freq. & Freq. Rel. \\
		\hline
		\multicolumn{1}{|c|}{1} & 0,16667 & 311 & 0,16694 & 274,01 & 0,16869 \\
		\multicolumn{1}{|c|}{2} & 0,16667 & 311 & 0,16694 & 267,74 & 0,16483 \\
		\multicolumn{1}{|c|}{3} & 0,16667 & 310 & 0,16640 & 271,14 & 0,16692 \\
		\multicolumn{1}{|c|}{5} & 0,16667 & 311 & 0,16694 & 273,42 & 0,16832 \\
		\multicolumn{1}{|c|}{6} & 0,16667 & 310 & 0,16640 & 262,80 & 0,16179 \\
		\multicolumn{1}{|c|}{7} & 0,16667 & 310 & 0,16640 & 275,25 & 0,16945 \\
		\hline
		\multicolumn{2}{|c|}{Diferen�a Quadr�tica} & \multicolumn{2}{c|}{0} & \multicolumn{2}{c|}{0,00004} \\
		\hline
	\end{tabular}	
	\label{tab:Fr_espinha}	
\end{table}		

O gr�fico (Figura \ref{fig:grafico_espinha}) mostra que a solu��o em Tempo Real depois de estabilizar teve um desempenho inferior � solu��o \textit{Offline}.
\begin{figure}[htb]
	\centering
		\includegraphics[width=1\textwidth]{../../../resultados/graficos/espinha.png}
	\caption{Gr�fico do Mapa E.}
	\label{fig:grafico_espinha}
\end{figure}

\clearpage
	\item Mapa F: Com seis salas de prioridades diferentes (Tabela \ref{tab:P_espinha_diff}) em formato de espinha de peixe (Figura \ref{fig:grafo_espinha}):

\begin{table}[!htb]
	\centering
	\caption{Prioridades do Mapa F.}
 	\begin{tabular}{|c|c|c|}
		\hline
		Sala & Prioridade\\
		\hline
			1& 5 \\
			2& 1 \\
			3& 2 \\
			5& 2 \\
			6& 1 \\
			7& 1 \\
		\hline
  \end{tabular}
  \label{tab:P_espinha_diff}
\end{table}

A seq��ncia de salas a serem visitadas gerada pelo \texttt{gerador} foi: 1 2 6 5 1 7 3 5.

As prioridades relativas finais da solu��o em Tempo Real foram pr�ximas �s prioridades relativas iniciais do mapa, como mostra a Tabela \ref{tab:Pr_espinha_diff}.		
 		
\begin{table}[!hbt]		
	\centering	
	\caption{Prioridades relativas do Mapa F.}	
	\begin{tabular}{|c|c|c|}	
		\hline
		Sala & \textit{Offline} & Tempo Real \\
		\hline
		1 & 0,41667 & 0,41297 \\
		2 & 0,08333 & 0,08386 \\
		3 & 0,16667 & 0,16624 \\
		5 & 0,16667 & 0,17418 \\
		6 & 0,08333 & 0,07849 \\
		7 & 0,08333 & 0,08427 \\
		\hline
	\end{tabular}	
	\label{tab:Pr_espinha_diff}	
\end{table}		
 		
A Tabela \ref{tab:Fr_espinha_diff} mostra que as freq��ncias relativas da solu��o \textit{Offline} foram diferentes das prioridades relativas iniciais do mapa, visitando as salas 1 e 5 o dobro de vezes que as demais salas, pois na seq��ncia determinada pelo \texttt{gerador} as salas 1 e 5 aparecem duas vezes cada uma. Todavia as freq��ncias relativas da solu��o em Tempo Real foram pr�ximas �s prioridades relativas iniciais do mapa. A Tabela \ref{tab:Fr_espinha_diff} mostra tamb�m que a solu��o Tempo Real foi melhor que a solu��o \textit{Offline} atrav�s da diferen�a quadr�tica.
 		
\begin{table}[!hbt]
	\centering	
	\caption{Freq��ncias Relativas do Mapa F.}	
	\begin{tabular}{cc|c|c|c|c|}	
		\cline{3-6}
		& & \multicolumn{2}{c|}{\textit{Offline}}& \multicolumn{2}{c|}{Tempo Real}\\
		\hline
		\multicolumn{1}{|c|}{Sala} & Prioridade Rel. & Freq. & Freq. Rel. & Freq. & Freq. Rel. \\
		\hline
		\multicolumn{1}{|c|}{1} & 0,41667 & 462 & 0,25000 & 585,01 & 0,35364 \\
		\multicolumn{1}{|c|}{2} & 0,08333 & 231 & 0,12500 & 167,40 & 0,10119 \\
		\multicolumn{1}{|c|}{3} & 0,16667 & 231 & 0,12500 & 283,61 & 0,17144 \\
		\multicolumn{1}{|c|}{5} & 0,16667 & 462 & 0,25000 & 294,42 & 0,17798 \\
		\multicolumn{1}{|c|}{6} & 0,08333 & 231 & 0,12500 & 156,95 & 0,09488 \\
		\multicolumn{1}{|c|}{7} & 0,08333 & 231 & 0,12500 & 166,87 & 0,10087 \\
		\hline
		\multicolumn{2}{|c|}{Diferen�a Quadr�tica} & \multicolumn{2}{c|}{0,04167} & \multicolumn{2}{c|}{0,00488} \\
		\hline
	\end{tabular}	
	\label{tab:Fr_espinha_diff}	
\end{table}		
O gr�fico (Figura \ref{fig:grafico_espinha_diff}) mostra que a solu��o em Tempo Real depois de estabilizar teve um desempenho inferior � solu��o \textit{Offline}.

\begin{figure}[!hbt]
	\centering
		\includegraphics[width=1\textwidth]{../../../resultados/graficos/espinha_diff.png}
	\caption{Gr�fico do Mapa F.}
	\label{fig:grafico_espinha_diff}
\end{figure}
	
	\item Mapa G: Com oito salas de prioridades diferentes (Tabela \ref{tab:P_ap}) simulando um apartamento real (Figura \ref{fig:grafo_ap}):

\begin{figure}[!hbt]
	\centering
		\includegraphics[width=0.8\textwidth]{../../imagens/mapa_ap.png}
	\caption{Grafo do Mapa G.}
	\label{fig:grafo_ap}
\end{figure}


\begin{table}[!hbt]
	\centering
	\caption{Prioridades do Mapa G.}
 	\begin{tabular}{|c|c|c|}
		\hline
		Sala & Prioridade\\
		\hline
		1 & 5 \\
		3 & 5 \\
		4 & 3 \\
		5 & 3 \\
		7 & 5 \\
		9 & 4 \\
		11& 4 \\
		12& 4 \\
		\hline
  \end{tabular}
  \label{tab:P_ap}
\end{table}

A seq��ncia de salas a serem visitadas gerada pelo \texttt{gerador} foi: 1 12 9 11 5 7 3 4.

As prioridades relativas finais da solu��o em Tempo Real foram pr�ximas �s prioridades relativas iniciais do mapa, como mostra a Tabela \ref{tab:Pr_ap}.		
 		
\begin{table}[!hbt]
	\centering	
	\caption{Prioridades relativas do Mapa G.}	
	\begin{tabular}{|c|c|c|}	
		\hline
		Sala & \textit{Offline} & Tempo Real \\
		\hline
		1 & 0,15152 & 0,15426 \\
		3 & 0,15152 & 0,15233 \\
		4 & 0,09091 & 0,08996 \\
		5 & 0,09091 & 0,08924 \\
		7 & 0,15152 & 0,14934 \\
		9 & 0,12121 & 0,12566 \\
		11& 0,12121 & 0,12133 \\
		12& 0,12121 & 0,11789 \\
		\hline
	\end{tabular}	
	\label{tab:Pr_ap}	
\end{table}		
 		
A Tabela \ref{tab:Fr_ap} mostra que as freq��ncias relativas da solu��o \textit{Offline} foram aproximadamente todas iguais, isso se deve ao fato de que a seq��ncia de salas a serem visitadas gerada pelo \texttt{gerador} manda o rob� visitar cada sala apenas uma �nica vez dentro do \textit{loop}, por�m as maiores diferen�as num�ricas s�o as das sala 4 e 5 que s�o aproximadamente 0,03. Na mesma tabela pode-se ver tamb�m que as freq�encias relativas da solu��o em Tempo Real foram pr�ximas �s prioridades relativas iniciais do mapa.	A Tabela \ref{tab:Fr_ap} mostra tamb�m que a solu��o em Tempo Real foi melhor que a solu��o \textit{Offline} atrav�s da diferen�a quadr�tica.
 		
\begin{table}[!hbt]
	\centering	
	\caption{Freq��ncias Relativas do Mapa G.}	
	\begin{tabular}{cc|c|c|c|c|}	
		\cline{3-6}
		& & \multicolumn{2}{c|}{\textit{Offline}}& \multicolumn{2}{c|}{Tempo Real}\\
		\hline
		\multicolumn{1}{|c|}{Sala} & Prioridade Rel. & Freq. & Freq. Rel. & Freq. & Freq. Rel. \\
		\hline
		\multicolumn{1}{|c|}{1} & 0,15152 & 247 & 0,12525 & 234,33 & 0,14970 \\
		\multicolumn{1}{|c|}{3} & 0,15152 & 246 & 0,12475 & 237,48 & 0,15172 \\
		\multicolumn{1}{|c|}{4} & 0,09091 & 246 & 0,12475 & 152,10 & 0,09717 \\
		\multicolumn{1}{|c|}{5} & 0,09091 & 246 & 0,12475 & 147,87 & 0,09447 \\
		\multicolumn{1}{|c|}{7} & 0,15152 & 246 & 0,12475 & 222,23 & 0,14197 \\
		\multicolumn{1}{|c|}{9} & 0,12121 & 247 & 0,12525 & 194,75 & 0,12442 \\
		\multicolumn{1}{|c|}{11}& 0,12121 & 247 & 0,12525 & 187,90 & 0,12004 \\
		\multicolumn{1}{|c|}{12}& 0,12121 & 247 & 0,12525 & 188,62 & 0,12050 \\
		\hline
		\multicolumn{2}{|c|}{Diferen�a Quadr�tica} & \multicolumn{2}{c|}{0,00446} & \multicolumn{2}{c|}{0,00016} \\
		\hline
	\end{tabular}	
	\label{tab:Fr_ap}	
\end{table}		

O gr�fico (Figura \ref{fig:grafico_ap}) mostra que a solu��o em Tempo Real depois de estabilizar teve um desempenho inferior � solu��o \textit{Offline}.

\begin{figure}[!hbt]
	\centering
		\includegraphics[width=1\textwidth]{../../../resultados/graficos/ap.png}
	\caption{Gr�fico do Mapa G.}
	\label{fig:grafico_ap}
\end{figure}

\subsection{An�lise Comparativa}
Foi calculada uma m�dia do Grau de Urg�ncia Total de cada solu��o em cada mapa com os �ltimos dez mil segundos (Tabela \ref{tab:media_final}); analisando essa m�dia nota-se que existe uma tend�ncia de que quanto maior o n�mero de salas mais eficiente ser� a solu��o \textit{Offline} em rela��o � solu��o em Tempo Real, como mostra o gr�fico na Figura \ref{fig:grafico_tendencia}.
\begin{table}[!hbt]
	\centering
	\caption{M�dia dos �ltimos Dez Mil Segundos.}
		\begin{tabular}{|c|c|c|}
			\hline			
			Mapa & & \\ 
			(N\# de Salas) & \textit{Offline} & Tempo Real \\
			\hline
			A (4) 			& 25,000 & 28,150 \\
			B (4) & 20,980 & 28,346 \\
			C (5) 					& 28,905 & 38,753 \\
			D (5)& 39,719 & 50,326 \\
			E (6)			& 58,527 & 72.622 \\
			F (6)& 55,557 & 66,711 \\
			G (8)	& 73,318 &100,217 \\
			\hline			
		\end{tabular}
	\label{tab:media_final}
\end{table}


\begin{figure}[!hbt]
	\centering
		\includegraphics{../../imagens/tendencia.png}
	\caption{Tend�ncia de Desempenho}
	\label{fig:grafico_tendencia}
\end{figure}
\end{itemize}
\clearpage

\section{Considera��es}
Analisando os testes pode-se constatar que:

\begin{itemize}
	\item Segundo o crit�rio de avalia��o de freq��ncia relativa, a solu��o em Tempo Real foi superior � solu��o \textit{Offline}, pois foi mais eficiente em cinco dos sete testes e na soma total das diferen�as quadr�ticas (0,10812 da solu��o \textit{Offline} contra 0,00652 da solu��o em Tempo Real);
	\item Segundo o crit�rio de avalia��o de Grau de Urg�ncia Total a solu��o \textit{Offline} foi melhor que a solu��o em Tempo Real em todos os testes realizados e o gr�fico da Figura \ref{fig:grafico_tendencia} mostrou que existe um tend�ncia de que quanto maior o n�mero de salas maior ser� a efici�ncia da solu��o \textit{Offline} em rela��o � solu��o em Tempo Real.
\end{itemize}

Era esperado que a solu��o \textit{Offline} tivesse um desempenho pior que a solu��o em Tempo Real no crit�rio de avalia��o de freq��ncia relativa, pois a avalia��o das seq��ncias de salas � visitar � baseada somente no Grau de Urg�ncia Total; e o desempenho das solu��es segundo o crit�rio de avalia��o de Grau de Urg�ncia Total mostra que o rob�, seguindo uma heur�stica simples, tem um desempenho inferior do que uma solu��o que determina a seq��ncia de salas a serem visitadas, baseada no Grau de Urg�ncia Total, antes de colocar o rob� no ambiente.

Por fim, o que determina qual solu��o deve ser utilizada depende da aplica��o. Se para a aplica��o a freq��ncia relativa for mais importante do que manter o Grau de Urg�ncia baixo, deve-se escolher a solu��o em Tempo Real, por�m se para a aplica��o for mais importante manter o Grau de Urg�ncia baixo deve-se escolher a solu��o \textit{Offline}, que foi baseada no Grau de Urg�ncia Total.

\chapter{Conclus�o e Trabalhos Futuros}
\label{conclusao}

CONCLUSAO

Para dar continuidade ao projeto o pr�ximo � explorar a utiliza��o de mais de um rob� para realizar a tarefa, cada rob� monitorar um conjunto de salas divididas por prioridade ou proximidade, ou todos os rob�s podem monitorar todas as salas etc.
%\appendix

%
% Para ajustar padr�o... C:\Program Files (x86)\MiKTeX 2.8\bibtex\bst\apalike\apalike.bst
%
%

%\bibliographystyle{apalike}
\bibliographystyle{plain}

\bibliography{../../bibliografia}
\addcontentsline{toc}{chapter}{Refer�ncias}

\end{document}


 