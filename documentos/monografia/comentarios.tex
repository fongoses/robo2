%==========================================================
% SCE=567 Projeto Supervisionado ou de Graduacao II
% 
% Capitulo 5: Comentarios sobre o Curso de Graduacao
% 
% Autor
% Heitor Luis Polidoro
% 
% Orientador
% Prof. Dr. Denis Fernando Wolf
%==========================================================

\chapter{Coment�rios sobre o Curso de Gradua��o}
\label{comentarios_sobre_curso_grad}
O curso de Bacharelado em Ci�ncias de Computa��o no ICMC/USP � excelente, contudo tenho algumas cr�ticas. Senti falta de mais op��es de disciplinas sobre intelig�ncia artificial cursei somente as disciplinas \textit{Intelig�ncia Artificial} e \textit{Introdu��o a Redes Neurais}. A disciplina \textit{Intelig�ncia Artificial} n�o correspondeu �s minhas espectativas pois eu esperava aprender conceitos de intelig�ncia artificial mas no curso foi ensinado mais a utilizar a linguagem de programa��o \textit{ProLog} do que conceitos de intelig�ncia artificial.

Uma outra cr�tica � com rela��o � postura do instituto quanto �s �nfases. N�o existe um programa de palestras com pessoas da �rea de cada �nfase, escolhi a �nfase de sistemas embarcados pela afinidade com a �rea e por n�o ter afinidade com as outras �nfases, mas n�o obtive nenhuma instru��o do que seria a �nfase, e o que existe de pesquisa dentro dela, entrei para a �rea de rob�tica durante o curso, pois tivemos mat�rias (\textit{Projeto e Implementa��o de Sistemas Embarcados I} e \textit{II}) que utilizamos a biblioteca e o simulador Player/Stage, e gostei. Caso, antes da escolha da �nfase, tivesse existido alguma palestra explicando mais sobre a mesma, teria sido mais f�cil a decis�o.
