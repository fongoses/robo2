%==========================================================
% Resumo
% 
% Autor
% Heitor Luis Polidoro
% 
% Orientador
% Prof. Dr. Denis Fernando Wolf
%==========================================================

\pagestyle{chapterst}

\chapter*{Resumo}
\label{Resumo}
A rob�tica m�vel � uma �rea de pesquisa que est� obtendo grande aten��o da comunidade cient�fica. O desenvolvimento de rob�s m�veis aut�nomos, que sejam capazes de interagir com o ambiente, aprender e tomar decis�es corretas para que suas tarefas sejam executadas com �xito � o maior desafio em rob�tica m�vel. O desenvolvimento destes sistemas inteligentes e aut�nomos consiste em uma �rea de pesquisa multidisciplinar considerada recente e extremamente promissora que envolve; por exemplo, intelig�ncia artificial, aprendizado de m�quina, estima��o estat�stica e sistemas embarcados. Dentro desse contexto, esse trabalho aborda o problema de navega��o e monitoramento de ambientes utilizando rob�s m�veis. Dada uma representa��o do ambiente (mapa topol�gico) e uma lista com urg�ncias de cada uma das regi�es do mapa, o rob� deve estimar qual o percurso mais eficiente para monitorar esse ambiente. Uma vez que a urg�ncia de cada regi�o n�o visitada aumenta com o tempo, o trajeto do rob� deve se adaptar a essas altera��es. Entre as diversas aplica��es pr�ticas desse tipo de algoritmo, destaca-se o desenvolvimento de sistemas de seguran�a m�veis inteligentes. 
