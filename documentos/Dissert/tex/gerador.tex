%==========================================================
% Capitulo x: Gerador.
% 
% Autor
% Heitor Luis Polidoro
% 
% Orientador
% Prof. Dr. Denis Fernando Wolf
%==========================================================

\pagestyle{chapterst}

\chapter{Gerador}
\label{gerador}

O programa chamado \texttt{gerador} � utilizado para gerar os candidatos a seq��ncia de salas �tima do mapa analisado. Esses candidatos s�o avaliados pelo programa \texttt{avaliador} e de acordo com a avalia��o o \texttt{gerador} ir� guardar a seq��ncia como poss�vel �tima, descartar ou continuar utilizando essa seq��ncia para gerar novas seq��ncia. 

\section{Funcionamento}

O programa \textit{gerador} utiliza uma classe (tipo de vari�vel) chamada \textit{Agente} contendo os seguintes atributos:

\begin{itemize}
	\item \texttt{vertice}: V�rtice no qual o agente se encontra no momento;
	\item \texttt{caminho}: Vetor de salas que guarda a seq��ncia de salas que o agente percorreu at� chegar no v�rtice atual;
	\item \texttt{avaliacao}: Avalia��o da seq��ncia de salas do agente;
	\item \texttt{tempo}: Tempo em segundos que o rob� levou para percorrer a seq��ncia de salas at� o v�rtice atual.
\end{itemize}

O \texttt{gerador} recebe apenas um par�metro como entrada: o nomo do mapa que quer achar a seq��ncia �tima. O programa inicia criando uma seq��ncia percorrendo as salas na ordem num�rica (Sala 1, sala 2..) e classificada como poss�vel �tima. A seguir o programa cria um \textit{Agente} em cada sala do mapa para explorar as possibilidades do rob� come�ar em cada sala, esses \textit{agentes} s�o organizados em uma fila de candidatos.

O la�o principal do programa consiste em retirar um \textit{agente} da fila e para cada sala

\section{Algoritmo}

\begin{algorithm}
\caption{\texttt{gerador}}
\label{alg:gerador}
\begin{algorithmic}[1]
\REQUIRE  
\ENSURE 
	\STATE 
	\REPEAT
		\FORALL {}
			\IF {}
				\RETURN $-aval$
			\ENDIF
		\ENDFOR
	\UNTIL 
\end{algorithmic}
\end{algorithm}

