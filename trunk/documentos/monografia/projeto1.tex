%==========================================================
% SCE=567 Projeto Supervisionado ou de Graduacao II
% 
% Apendice A: Atividades Desenvolvidas no Projeto de
% Graduacao I.
% 
% Autor
% Heitor Luis Polidoro
% 
% Orientador
% Prof. Dr. Denis Fernando Wolf
%==========================================================

\pagestyle{chapterst}

\chapter{Atividades Desenvolvidas no Projeto de Gradua��o I}
\label{atividades_projeto1}
No primeiro semestre de 2007 o autor desta monografia cursou a disciplina \emph{SCE 292:Projeto de Gradua��o I} e apresentou o trabalho intitulado \emph{Desenvolvimento de t�cnicas de monitoramento de ambientes internos utilizando-se rob�s m�veis} sob orienta��o do Prof. Dr. Denis Wolf. Para uma melhor compreens�o deste trabalho, que d� continuidade ao projeto iniciado no semestre anterior, as atividades j� realizadas s�o brevemente descritas neste ap�ndice.

\textbf{Estudo do ambiente de controle de rob�s m�veis Player/Stage}

Onde foi estudado a biblioteca e o simulador Player/Stage.

\textbf{Estudo e implementa��o do algoritmo de desvio de obst�culos}

Onde foi estudado e implementado o algoritmo de Campos Potenciais.

\textbf{Estudo e implementa��o dos algoritmos de planejamento de trajet�ria}

Onde foi analisado os diferentes algoritmos de melhor caminho (Busca em Profundidade, Busca em Largura, Dijsktra e Ants), e foi implementado o Algoritmo \textit{Ants} pela facilidade de paralelismo

\textbf{Desenvolvimento e implementa��o dos algoritmos de determina��o das �reas que devem ser visitadas}

Onde foi desenvolvido e implementado dois tipos de algoritmos para uma primeira an�lise do uso da rob�tica para monitoramento de ambientes internos.

\textbf{Integra��o e valida��o dos algoritmos implementados no simulador Stage}

Onde foi testado os algoritmos desenvolvidos, verificando que ambos obedeceram � prioridade das salas, visitando com maior freq��ncia as salas de maior prioridade e visitando com menor freq��ncias as salas de menor prioridade.