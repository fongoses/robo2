%==========================================================
% Projeto de Mestrado
% 
% Resumo.
% 
% Autor
% Heitor Luis Polidoro
% 
% Orientador
% Prof. Dr. Denis Fernando Wolf
%==========================================================

\pagestyle{abstractst}

\chapter*{Resumo}
\label{resumo}
A robótica móvel é uma área de pesquisa que está obtendo grande atenção da comunidade científica. O desenvolvimento de robôs móveis autônomos, que sejam capazes de interagir com o ambiente, aprender e de tomar decisões corretas para que suas tarefas sejam executadas com êxito é o maior desafio em robótica móvel. O desenvolvimento destes sistemas inteligentes e autônomos consiste em uma área de pesquisa multidisciplinar, considerada recente e extremamente promissora que envolve: inteligência artificial, aprendizado de máquina, estimação estatística e sistemas embarcados, por exemplo. Dentro desse contexto, esse trabalho aborda o problema de navegação e monitoramento de ambientes utilizando robôs móveis. Dado uma representação do ambiente (mapa topológica) e uma lista com urgências de cada uma das regiões do mapa, o robô deve estimar qual o percurso mais eficiente para monitorar esse ambiente. Uma vez que a urgência de cada região não visitada aumenta com o tempo, o trajeto do robô deve-se adaptar a essas alterações. Entre as diversas aplicações práticas desse tipo de algoritmo, destaca-se o desenvolvimento de sistemas de segurança móveis inteligentes. Experimentos com diferentes algoritmos em diferentes ambientes são apresentados.


