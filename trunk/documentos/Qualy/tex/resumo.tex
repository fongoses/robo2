%==========================================================
% Resumo.
% 
% Autor
% Heitor Luis Polidoro
% 
% Orientador
% Prof. Dr. Denis Fernando Wolf
%==========================================================

\pagestyle{abstractst}

\chapter*{Resumo}
\label{resumo}
A rob�tica m�vel � uma �rea de pesquisa que est� obtendo grande aten��o da comunidade cient�fica. O desenvolvimento de rob�s m�veis aut�nomos, que sejam capazes de interagir com o ambiente, aprender e de tomar decis�es corretas para que suas tarefas sejam executadas com �xito � o maior desafio em rob�tica m�vel. O desenvolvimento destes sistemas inteligentes e aut�nomos consiste em uma �rea de pesquisa multidisciplinar, considerada recente e extremamente promissora que envolve: intelig�ncia artificial, aprendizado de m�quina, estima��o estat�stica e sistemas embarcados, por exemplo. Um dos problemas fundamentais na �rea de rob�tica � a navega��o, que consiste em determinar o local para onde o rob� deve se mover, o melhor caminho at� esse local e como chegar at� esse local de forma segura, evitando obst�culos. Esse projeto de pesquisa prop�e a compara��o de algoritmos de monitoramento de ambientes internos, no qual o ambiente � dividio em �reas de interesse e cada �rea recebe uma prioridade, o rob� ent�o deve percorrer o ambiente visitando essas �reas de maneira eficiente, de acordo com alguns crit�rios propostos no trabalho, levando em considera��o a prioridade de cada local a ser visitado e o tempo decorrido desde a �ltima visita a cada regi�o do ambiente.
