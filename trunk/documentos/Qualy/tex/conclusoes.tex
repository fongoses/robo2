%==========================================================
% SCE=567 Projeto Supervisionado ou de Graduacao II
% 
% Capitulo 4: Conclusoes e Trabalhos Futuros.
% 
% Autor
% Heitor Luis Polidoro
% 
% Orientador
% Prof. Dr. Denis Fernando Wolf
%==========================================================

\chapter{Conclus�es e Trabalhos Futuros}
\label{conclusoes_trabalhos_futuros}
Analisando os resultados obtidos descritos na se��o \ref{resultados_obtidos}, podemos concluir que o desempenho do algoritmo 1 foi inferior em todos os testes, obtendo sempre uma pontua��o menor e um grau de urg�ncia total superior aos outros algoritmos. Os algoritmos 2 e 3 alternaram seu desempenho, sendo o algoritmo 2 melhor na pontua��o do mapa 1, mas praticamente empatado com o algoritmo 3 no mapa 2 como mostra a tabela \ref{tab:comp_pontos}.
\begin{table}
\begin{center}\begin{tabular}{|c|c|c|c|}
\hline
  &  Algoritmo 1 &  Algoritmo 2 & Algoritmo 3 \\ 
\hline 
 Mapa 1 &  227 &  385 & 288 \\ 
\hline 
 Mapa 2 &  90 &  124 & 122 \\ 
\hline 
\end{tabular}
\caption{Compara��o da Pontua��o dos Algoritmos nos Mapas.}
\label{tab:comp_pontos}
\end{center}\end{table}

Analisando melhor esses dados, podemos verificar que no Mapa 1 (Figura \ref{fig:mapa_rep}), o Algoritmo 2 visitou com maior freq��ncia as salas de maior prioridade (Salas 1 e 3), como mostra na Tabela \ref{tab:freq_pontos_2_1}. Por�m, as salas 7, 11 e 12 n�o foram visitadas. E as salas 6 e 9 que est�o no caminho foram visitadas com muita freq��ncia. No caso do Algoritmo 3 o rob� visitou todas as salas pelo menos uma vez, entretanto visitou com freq��ncia semelhante salas com prioridades opostas, como as salas 1, 2 e 3 (Tabela \ref{tab:freq_pontos_3_1}.

Ao analisarmos os gr�ficos obtemos conclus�es diferentes. Pois no gr�fico da Figura \ref{fig:grafico_rep_30m} mostra claramente um desempenho melhor do Algoritmo 3, mantendo o Grau de Urg�ncia Total sempre inferior aos outros algoritmos. No entanto, no gr�fico da Figura \ref{fig:grafico_andar_30m} os algoritmos 2 e 3 tem um desempenho equivalente, e no final do gr�fico existe uma mudan�a brusca.

Com isso, � poss�vel afirmar que o desempenho dos algoritmos desenvolvidos variam de acordo com a necessidade da aplica��o.

Como continua��o desse projeto, pretende-se desenvolver um projeto semelhante utilizando uma equipe de rob�s m�veis, criando novos algoritmos de determina��o de trajet�ria, alternando entre se haver� comunica��o entre os rob�s, se os rob�s ser�o totalmente aut�nomos ou obedecer�o a um mestre.

