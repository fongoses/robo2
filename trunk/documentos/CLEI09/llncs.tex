% This is LLNCS.DEM the demonstration file of
% the LaTeX macro package from Springer-Verlag
% for Lecture Notes in Computer Science, version 1.1
\documentstyle{llncs}
 
%
\begin{document}

\title{Desenvolvimento de Rotas para Monitorar Ambientes Internos}

\author{Heitor L. Polidoro\inst{1} and Denis F. Wolf\inst{1}}

\institute{Instituto de Ci\^{e}ncias Matem\'{a}ticas e de Computa\c{c}\~{a}o -- Universidade de S\~{a}o Paulo
  (USP), S\~{a}o Carlos-SP, Brazil}

\maketitle

\begin{abstract}
Um dos principais desafios na \'{a}rea de rob\^{o}s m\'{o}veis \'{e} o desenvolvimento de sistemas aut\^{o}nomos, que sejam capazes de interagirem com o ambiente, aprenderem e tomarem decis\~{o}es corretas para que suas tarefas sejam executadas com \^{e}xito. Dentro desse contexto, esse trabalho aborda o problema de navega\c{c}\~{a}o e monitoramento de ambientes utilizando rob\^{o}s m\'{o}veis. Dado uma representa\c{c}\~{a}o do ambiente (mapa topol\'{o}gico) e uma lista com urg\^{e}ncias de cada uma das regi\~{o}es do mapa, o rob\^{o} deve estimar qual o percurso mais eficiente para monitorar esse ambiente. Uma vez que a urg\^{e}ncia de cada regi\~{a}o n\~{a}o visitada aumenta com o tempo, o trajeto do rob\^{o} deve se adaptar a essas altera\c{c}\~{o}es. Entre as diversas aplica\c{c}\~{o}es pr\'{a}ticas desse tipo de algoritmo, destaca-se o desenvolvimento de sistemas de seguran\c{c}a m\'{o}veis inteligentes.

\end{abstract}
%
\end{document}
